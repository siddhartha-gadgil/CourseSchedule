\subsection{A. K. Nandakumaran}

We have demonstrated the asymptotic analysis of a semi-linear optimal control problem posed on a smooth oscillating boundary domain in the present paper. We have considered a more general oscillating domain than the usual “pillar-type” domains. Consideration of such general domains will be useful in more realistic applications like circular domain with rugose boundary. We study the asymptotic behavior of the problem under consideration using a new generalized periodic unfolding operator. Further, we have studied the homogenization of a non-linear optimal control problem and such non-linear problems are limited in the literature despite the fact that they have enormous real-life applications. Among several other technical difficulties, the absence of a enough criteria for the optimal control is one of the most attention-grabbing issues in the current setting. We also obtain corrector results in this paper.  We considered an optimal control problem posed on a domain with a highly oscillating smooth boundary where the controls are applied on the oscillating part of the boundary. There are many results on domains with oscillating boundaries where the oscillations are pillar-type (non-smooth) while the literature on smooth oscillating boundary is very few. In this article, we use appropriate scaling on the controls acting on the oscillating boundary leading to different limit control problems; namely, boundary optimal control and interior optimal control problem. In the last part of the article, we visualize the domains as a branched structure, and we introduce unfolding operators to get contributions from each level at every branch. 


\subsection{Apoorva Khare}

(1) A comprehensive, 40+ page paper treating distance matrices of trees. (2) An 80+ page survey on matrix positivity. (3) Extending results of Loewner, Horn, Cauchy, and Frobenius to new symmetric function identities and positivity conditions. (Among other works.)


\subsection{Arvind Ayyer}

Multispecies versions of random juggling models were studied and exact stationary distributions were obtained using combinatorial techniques [1]. For one variant, ultrafast convergence to stationarity was proved.  A new totally asymmetric exclusion process with two species of particles on a finite one-dimensional lattice in contact with reservoirs was introduced, where the second-class particle is impermeable to the right reservoir [2]. The stationary distribution was obtained using a matrix ansatz and the nonequilibrium phase diagram was computed in the thermodynamic limit. This was later extended to multiple species of particles and the phase diagram was computed.  A new interacting particle system called the Hilbert-Galton board was introduced and various theorems about this process were proved [3].  Random walks on noncommutative finite rings generated by both addition and multiplication were studied. Results were proven about the spectrum, stationary distribution and mixing time [4].  A new class of factorizations about characters of the general linear group, when the variables were specialised, were proved [5]. Somewhat surprisingly, the resulting objects are themselves characters of other classical groups. Some applications to enumerations of plane partitions and alternating sign matrices was given.  The combinatorics of a large class of cyclic orders were studied [6]. In particular, the enumeration of circular extensions of these cyclic orders was related to the enumeration of a family of polytopes made popular by R. Stanley.  Extending earlier work, odd and chiral representations were studied for all Coxeter groups [7].   [1] A. Ayyer, C. Finn and D. Roy, Matrix product solution of a left-permeable two-species asym- metric exclusion process, Phys. Rev. E, 97 no. 1 (2018) 012151, 10pp. [2] A. Ayyer, J. Bouttier, S. Corteel, S. Linusson and F. Nunzi, Bumping sequences and multi- species juggling, Advances in Applied Mathematics, 98 (2018) 100–126. [3] A. Ayyer and S. Ramassamy, The Hilbert-Galton board, Latin American Journal of Probability and Mathematical Statistics (ALEA), 15 no. 2 (2018), 755–774. [4] A. Ayyer and P. Singla, Random motion on finite rings, II: Noncommutative rings,  arXiv:1807.04082. [5] A. Ayyer and R. E. Behrend, Factorization theorems for classical group characters, with applications to alternating sign matrices and plane partitions, arXiv:1804.04514 [6]  A. Ayyer, M. Josuat-Vergès and S. Ramassamy, Extensions of partial cyclic orders and consecutive coordinate polytopes,  arXiv:1803.10351. [7]  A. Ayyer, A. Prasad and S. Spallone, Macdonald trees and determinants of representations for finite Coxeter groups, arXiv:1812.00608.  


\subsection{Basudeb Datta}

We have shown that the boundary of the pseudorhombicuboctahedron is the only semi-equivelar map on the 2-sphere which is not vertex-transitive.   As a consequence, we show that each semi-equivelar map on the real projective plane is vertex-transitive.


\subsection{Dilip P Patil}

(1) During my sabbatical leave in IIT Bombay I have written the following two manuscripts (jointly with Prof. J. K. Verma):  [1] (Jointly with Kriti Goel and J. K. Verma ) Nullstellens\"atze and Applications, to appear in Conference Proceedings Published by Springer.   In this expository paper, we present simple proofs of the Classical, Real, Projective  and Combinatorial Nullstellens\"atze. Several applications are also presented such as a classical theorem of Stickelberger for solutions of polynomial equations in terms of eigenvalues of commuting operators, construction of a principal ideal domain which is not Euclidean, Hilbert's $17^{th}$ problem,  the Borsuk-Ulam theorem in topology and solutions of the conjectures of Dyson, Erd\"{o}s and Heilbronn.  [2]  (Jointly wit  J. K. Verma) Rational points and generalized trace forms on a finite algebra over a real closed field, to appear in Conference Proceedings Published by Springer.   The core of this article is the equality of the number of $K$-rational points  with the signature of the trace form of a finite $K$-algebra over a real closed field $K$. The main tools are symmetric bilinear forms, hermitian forms, trace forms, generalized trace forms and their types and signatures.  Further, we prove a criterion for the existence of $K$-rational points by using generalized trace forms. As an application we prove the Pederson-Roy-Szpirglas theorem  about counting common real zeros of  real polynomial equations. We also prove deduce this theorem as a special case of a similar classical theorem for real closed fields.  [3] On the connectedness of the fiber product of two connected finte free affine schemes. Preprint.  This article grew out while studying Burnside algebra of finite abelian groups.             


\subsection{E. K. Narayanan}

Study of Heckman-Opdam hypergeometric functions associated to root systems of type $BC,$ where the associated multiplicity function was allowed to assume negative values, was continued.


\subsection{Gautam Bharali}

Further work was undertaken to investigate a weak form of negative curvature --- known as visibility --- exhibited by a range of domains in $\mathbb{C}^n$ equipped with the Kobayashi distance. The class of domains that are visibility domains with respect to the Kobayashi distance was significantly broadened in the preprint [1]. Furthermore, two Wolff--Denjoy-type theorems were established for the latter class of domains.  [1] Gautam Bharali and Anwoy Maitra, A weak notion of visibility, a family of examples, and Wolff--Denjoy theorems.


\subsection{Govindan Rangarajan}

Detection of a causal relationship between two or more sets of data is an important problem across various scientific disciplines. Test statistics based on Granger causality ignore the effect of practical measurement impairments such as subsampling, additive noise, and finite sample effects. We modelled [1] the problem of detecting a causal relationship between two time series as a binary hypothesis test with the null and alternate hypotheses corresponding to the absence and presence of a causal relationship, respectively. We derived the distribution of the test statistic under the two hypotheses and show that measurement impairments can lead to suppression of a causal relationship between the signals, as well as false detection of a causal relationship, where there is none. We also used the derived results to propose two alternative test statistics for causality detection. These detectors are analytically tractable, which allowed us to design the detection threshold and determine the number of samples required to achieve a given missed detection and false alarm rate.   Prostate-associated gene 4 (PAGE4) is an intrinsically disordered protein implicated in prostate cancer. Experiments have shown that PAGE4 expands upon hyperphosphorylation and that this expansion is localized to its N-terminal half. To understand the interactions underlying this structural transition, we performed [2] molecular dynamics simulations using Atomistic AWSEM, a multi-scale molecular model that combines atomistic and coarse- grained simulation approaches. Our simulations show that electrostatic interactions drive transient formation of an N-terminal loop, the destabilization of which accounts for the dramatic change in size upon hyperphosphorylation. Finally, we construct a mechanism-based mathematical model that allows us to capture the interactions of different phosphoforms of PAGE4 with AP-1 and its downstream target, the androgen receptor (AR)—a key therapeutic target in prostate cancer. Our model predicts intracellular oscillatory dynamics indicating phenotypic heterogeneity in an isogenic cell population. Thus, conformational switching of PAGE4 may potentially affect the efficiency of therapeutically targeting AR activity.   We proposed [3] an application of the microtremor (ambient noise) H/V spectral ratio technique to identify significant rheological boundaries at shallow depths, estimate thickness of both lignite bearing Tertiary sedimentary sequence and late Cretaceous Deccan basalt flows and comprehend basinal geometry of Umarsar Basin (Babia syncline). Forty-six stations were gauged in a grid format at ~250 m resolution during the microtremor survey. We detected three rheological interfaces. Finally, we proposed the microtremor H/V spectral ratio technique as a tool to develop economical borehole plan with realistic reserve estimate and a step forward towards rapid economical assessment covering large mining lease areas.   [1] IEEE Trans Signal Proc. (2018) [2] J. Mol. Biol (2018) [3] J. Appl. Geophys. (2018)


\subsection{Manjunath Krishnapur}

Working on certain problems on nodal sets of Gaussian random functions and in random matrix theory.


\subsection{Mrinal Kanti Ghosh}

We analyze an augmented political business cycle model taking into account the effect of employment creation decisions by the ruling party jointly on inflation and growth. The objective is to maximize voter support in the next election that depends on the rate of unemployment as well as that of growth and inflation. We allow for randomness in the New Keynesian Phillips Curve (NKPC) model for the relationship between  inflation and unemployment as well as in a benchmark labour productivity function for   analyzing the growth rate. We provide explicit solution paths of the affine Markov   control problem that results from our formulation. We also provide numerical illustrations with plausible parametric configurations to generate more insight into our model. [1] G. K. Basak, M. K. Ghosh and D. Mukherjee, A Stochastic Model with Inflation, Growth and Technology  for the Political Business Cycle, J. Computational Economics, 53 (2019), 125-140.


\subsection{Pooja Singla}

In [1], We study some irreversible Markov chains on finite commutative rings randomly generated using both addition and multiplication. We describe the eigenvalues and their multiplicities. We also prove recursive formulas for stationary distribution. We extend this work to finite non-commutative rings in [2].   In [3], the Gelfand-Whittaker model for $GL_n(O)$ and $SL_n(O)$,where $O$ is a  compact discrete valuation ring is studied. We proved that this model is multiplicity free and describe its constituents.   In [4] We describe the abscissa of convergence of representation zeta function of $SL_2(O)$ for a compact discrete valuation ring $O$ with even characteristic. For this case, we also give construction of continuous primitive irreducible representations and study group algebra isomorphism problem.   [1]. Random motion on finite rings, I: commutative rings, Arvind Ayyer and Pooja Singla, to appear in Algebras and Representation Theory.  [2]. Random motion on finite rings, II: Non-commutative rings; Arvind Ayyer and Pooja Singla, preprint.   [3]. A multiplicity one theorem for groups of type $A_n$ over discrete valuation rings, Shiv Prakash Patel and Pooja Singla, preprint.  [4]. Representation Growth of Special Compact Linear Groups of Order Two,  M. Hassain and Pooja Singla, preprint. 


\subsection{R. Venkatesh}

By improving some earlier work of Felikson, Retakh and Tumarkin, a classification of all the regular subalgebras of Affind Kac-Moody Lie algebras have been obtained. By continuing this classification problem to Extend Affine Lie Algebras which are natural generalizaion of Affine Kac-Moody algebras, a classification of all the maximal closed sub-root systems of Affine reflection systems has also been done. The question of when two tensor products of irreducible integrable representations would be isomorphic to another for Borcherds-KacMoody algebras has been investigated.


\subsection{SANCHAYAN SEN}

In [1], the critical behavior of the component sizes for the configuration model has been studied when the tail of the degree distribution of a uniform vertex is a regularly-varying function with exponent $\tau -1$, where $\tau\in (3, 4)$.   In [2], the metric space scaling limits of the minimal spanning trees of both the random (simple) 3-regular graph and the 3-regular configuration model have been established, and it has been shown that both these models belong to the universality class identified in the work of Addario-Berry, Broutin, Goldschmidt, and Miermont [Annals of Probability, 2017].   [1] Heavy-tailed configuration models at criticality.  Authors: Souvik Dhara, Remco van der Hofstad, Johan van Leeuwaarden, Sanchayan Sen.  Annales de l'Institut Henri Poincaré (B) Probabilités et Statistiques. Accepted in March 2019.  [2] Geometry of the minimal spanning tree of a random 3-regular graph.  Authors: Louigi Addario-Berry, Sanchayan Sen ArXiv preprint. Announced in October 2018.   


\subsection{Siddhartha Gadgil}

We refined our previous results where, as part of the PolyMath 14 project, we had classified homogeneous length functions on groups. In particular we showed that quasi-homogeneous length functions on groups with vanishing commutator length (such as solvable groups) are bounded distance from pullbacks from Abelianizations.  We also replicated and made robust the computer-assisited proof we had found in the course of the above mentioned PolyMath project and investigated general principles involved. We further developed methods for computer assisted theorem proving.


\subsection{Soumya Das}

We continue our investigations on analytic aspects of Automorphic forms.


\subsection{Srikanth K. Iyer}

The random connection model is studied in the connectivity regime. Strong law results for the critical scaling parameter required to eliminate isolated nodes is derived. A sufficient condition for the graph to be connected with high probability is obtained. Other results include strong law results for the maximum and minimum vertex degrees [1]. The capacity of a wireless cellular network is defined and an exact characterization (non-asymptotic) of this natural capacity metric is derived. It is shown that that the capacity increases polynomially with the base station (BS) density in the low BS density regime and then scales inverse exponentially with the increasing BS density [2]. A model of merge and split group dynamics, also called fission-fusion dynamics, for heterogeneous populations that contain two types (or species) of individuals is proposed and analysed. It is predicted using the analytical model that there is a critical group size below which groups are more likely to be homogeneous and contain the abundant type or species. Despite the propensity of heterogeneous groups to split at higher rates, we find that groups are more likely to be heterogeneous but only above the critical group size. Monte Carlo simulation of the model show excellent agreement with these analytical model results. The implications of our results to empirical studies on flocking systems are discussed [3].


\subsection{Subhojoy Gupta}

The main focus of my research during this period was in developing a correspondence between meromorphic quadratic differentials and certain geometric structures on punctured Riemann surfaces. Some of this work arose out of an effort to understand how such ``meromorphic" geometric structures arise in degenerations, in the context of hyperbolic structures. Other collaborative projects during this period that resulted in papers include a project with Basudeb Datta (IISc), on understanding geometric tilings of the hyperbolic plane, and work with Indranil Biswas (TIFR Mumbai) and Sorin Dumitrescu (CNRS, France) that concerns branched projective structures on Riemann surfaces.   Papers written or published during this period:  [1]   (with Michael Wolf)  Meromorphic quadratic differentials and measured foliations on a Riemann surface, Mathematische Annalen, Vol. 373 1-2,  (February 2019), pg. 73-118.  [2]  Limits of harmonic maps and crowned hyperbolic surfaces, accepted for publication , Transactions of the AMS,  available at  https://doi.org/10.1090/tran/7777.   [3]  Holomorphic quadratic differentials in Teichmüller theory,  to appear, Handbook of Teichmüller theory Vol VII, available at  http://arxiv.org/abs/1902.06406.  [4] (with Basudeb Datta)  Uniform tilings of the hyperbolic plane,  submitted, available at http://arxiv.org/abs/1806.11393.  [5] (with Indranil Biswas and Sorin Dumitrescu) Branched projective structures on a Riemann surface and logarithmic connections, submitted, available at http://arxiv.org/abs/1808.04555.  


\subsection{Sundaram Thangavelu}

We have obtained sparse bounds for the lacunary maximal function associated to spherical means on the Heisenberg group [1]. As a consequence we have obtained an analogue of a result of C. P. Calderon for the lacunary spherical maximal function on Euclidean spaces.


\subsection{Tirthankar Bhattacharyya}

My research for the relevant period has been on Toeplitz operators.  An easily computable and simple condition which is necessary and sufficient for a commuting tuple of contractions to possess a non-zero Toeplitz operator has been found.  This condition is just that the adjoint of the product of the contractions is not pure. On one hand, this brings out the importance of the product of the contractions and on the other hand, the non-pureness turns out to be equivalent to the existence of a pseudo-extension to a tuple of commuting unitaries. A commutant pseudo-extension theorem has been obtained by studying the unique canonical unitary pseudo-extension of a tuple of commuting contractions. Also, with the help of a special completely positive map, a different proof of the existence of the unique canonical unitary pseudo-extension is given. On a different note, an extensive theory of Toeplitz operators on the symmetrized bidisc has been developed. 


\subsection{Vamsi Pritham Pingali}

I worked on three themes : 1) Connections on holomorphic vector bundles : Along with I. Biswas [1], I studied the differential geometry of parabolic ample bundles and connected it with the known algebro-geometric results.  I introduced a vector bundle Monge-Ampere equation in a preprint [2] and proved some existence results for it. 2) Coupled PDE on Kahler manifolds : Along with Ved Datar, I studied coupled cscK equations and provided a moment map interpretation in a preprint [3]. 3) Interpolation : D. Varolin and I [4] provided examples and counterexamples of affine algebraic interpolating manifolds.  [1] I. Biswas and V. Pingali. Metric properties of parabolic ample bundles, International Mathematical Research Notices, Nov 2018 (published online - doi:10.1093/imrn/rny259). [2] V. Pingali. A vector bundle version of the Monge-Ampere equation, arXiv : 1804.03934, Apr 2018. [3] V. Datar and V. Pingali. On coupled constant scalar curvature Kahler metrics (with Ved Datar), arXiv : 1901.10454, Jan 2019. [4] V. Pingali and D. Varolin. Nonuniformly flat affine algebraic hypersurfaces, arXiv : 1810.00895, Oct 2018. (Accepted for publication in Nagoya Mathematical Journal -  https://doi.org/10.1017/nmj.2019.2)