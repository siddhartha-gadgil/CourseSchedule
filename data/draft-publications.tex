
\subsection{Papers in Journals}

\begin{enumerate}
\item S Aiyappan, A. K. Nandakumaran and Abu Sufian: Asymptotic analysis of Boundary Optimal Control Problem on a General Branched Structure, \emph{Mathematical Methods in the Applied Sciences} {\bf 42} (2019. DOI: 10.1002/mma.5748), 6407-6434, DOI 10.1002/mma.5748.
\item K. Murali, A. K. Nandakumaran, Turgurt Durduran, Hari M Varma: Recovery of the diffuse correlation spectroscopy data-type from speckle contrast measurements: towards low-cost, deep-tissue blood flow measurements, \emph{Biomedical Optics Express 5395} {\bf 10} (2019).
\item S Aiyappan, A. K. Nandakumaran and Ravi Prakash: Locally Periodic Unfolding Operator for Highly Oscillating Rough Domains, \emph{Annali di Matematica Pura ed Applicata} {\bf 198} (2019), 1931-1954, DOI 10.1007/s10231-019-00848-7.
\item S Aiyappan, A. K. Nandakumaran and Ravi Prakash: Semi-linear optimal control problem on a smooth oscillating domain, \emph{Communications in Contemporary Mathematics} {\bf } (2019), 1-26, DOI 10.1142/S0219199719500299.
\item S Aiyappan, Editha C. Jose, Ivy Carol B. Lomerio and A. K. Nandakumaran: Control Problem on a Rough Circular Domain and Homogenization, \emph{Asymptotic Analysis} {\bf 115} (2019), 19-46, DOI 10.3233/ASY-191526.
\item M. Balodi, A. Banerjee, S. Ray: Cohomology of modules over H-categories and co-H-categories, to appear in \emph{Canadian Journal of Mathematics}, \url{https://doi.org/10.4153/S0008414X19000403}.
\item M. Balodi, A. Banerjee, S. Ray: Entwined modules over linear categories and Galois extensions, to appear in \emph{Israel Journal of Mathematics (Accepted)}, DOI (accepted), \url{https://arxiv.org/abs/1901.00323}.
\item A. Banerjee: On Noetherian schemes over $(\mathcal C,\otimes,1)$ and the category of quasi-coherent sheaves, to appear in \emph{Indiana University Mathematics Journal}, \url{http://www.iumj.indiana.edu/IUMJ/Preprints/8267.pdf}.
\item A. Banerjee: On differential torsion theories and rings with several objects, \emph{Canadian Math Bulletin} {\bf 62} (2019), 703--714, \url{https://www.cambridge.org/core/journals/canadian-mathematical-bulletin/article/on-differential-torsion-theories-and-rings-with-several-objects/574D81145FAAAD6B7C5648253FA41833}.
\item Khare, Apoorva and Rajaratnam, Bala: Probability inequalities and tail estimates for metric semigroups, \emph{Advances in Operator Theory (Special issue in honor of Rajendra Bhatia)} {\bf 5} (2020), 779-795, DOI 10.1007/s43036-020-00048-8, \url{https://link.springer.com/article/10.1007%2Fs43036-020-00048-8}.
\item Ghosh, Subhajit: Total Variation Cutoff for the Transpose Top-2 with Random Shuffle, to appear in \emph{Journal of Theoretical Probability}, DOI 10.1007/s10959-019-00945-6, \url{https://link.springer.com/article/10.1007%2Fs10959-019-00945-6}.
\item Datta, B. and Gupta, S.: Semi-regular tilings of the hyperbolic plane \label{bdsg_2019}, to appear in \emph{Discrete Comput Geom}, DOI https://doi.org/10.1007/s00454-019-00156-0, \url{https://doi.org/10.1007/s00454-019-00156-0}.
\item Datta, B. and Sarkar, S.: Equilibrium triangulations of some quasitoric 4-manifolds, \emph{Southeast Asian Bull. Math.} {\bf 44} (2020), 57--78.
\item Datta, B. and Maity, D.: Correction to: Semi-equivelar and vertex-transitive maps on the torus, \emph{Beitr Algebra Geom} {\bf 61} (2020), 187--188.
\item Martin Kreuzer and Dilip P. Patil: COMPUTATIONAL ASPECTS OF BURNSIDE RINGS PART II: IMPORTANT MAPS, to appear in \emph{Beiträge zur Algebra und Geometrie / Contributions to Algebra and Geometry}.
\item Gadadhar Misra: Professor, \emph{Adv. Math.} {\bf 351} (2019), 1105 - 1138, \url{https://www.sciencedirect.com/science/article/abs/pii/S0001870819302609}.
\item Bharali, Gautam and Maitra, Anwoy: A weak notion of visibility, a family of examples, and Wolff--Denjoy theorems, to appear in \emph{Annali della Scuola Normale Superiore di Pisa}, DOI 10.2422/2036-2145.201906\_007, \url{http://annaliscienze.sns.it/index.php?page=ForthcomingArticles}.
\item Maitra, Anwoy: On the continuous extension of Kobayashi isometries, \emph{Proc. Amer. Math. Soc.} {\bf 148} (2020), 3437--3451, \url{https://www.ams.org/journals/proc/2020-148-08/S0002-9939-2020-15038-1/}.
\item Vishnu Priya, N. and Senthilvelan, M. and  Rangarajan, G. and Lakshmanan, M: On symmetry preserving and symmetry broken bright, dark and antidark soliton solutions of nonlocal nonlinear Schrödinger equation, \emph{Physics Letters A} {\bf 383} (2019), 15-26.
\item Kashyap, G. A. R. S. R. K. and Bapat, D. and Das, D. and Gowaikar, R. D. and Amritkar, R. E. and Rangarajan, G. and Ravindranath, V. and Ambika, G: Synapse loss and progress of Alzheimer’s disease - A network model, \emph{Scientific Reports} {\bf 9} (2019), 6555.
\item Vishnu Priya, N. and Senthilvelan, M. and Rangarajan, G.: On the role of four-wave mixing effect in the interactions between nonlinear modes of coupled generalized nonlinear Schrodinger equation, \emph{Chaos} {\bf 29} (2019), 123135.
\item Sarkar, Amar Deep and Verma, Kaushal: A submultiplicative property of the Carath\'{e}odory metric on planar domains, \emph{Proc. Indian Acad. Sci. (Math. Sci.)} {\bf } (), \url{https://www.ias.ac.in/article/fulltext/pmsc/130/0035}.
\item Frauke Bleher, Ted Chinburg, Ralph Greenberg, Mahesh Kakde, George Pappas, Romyar Sharifi, Martin Taylor: Higher Chern Classes in Iwasawa Theory, \emph{American Journal of Mathematics} {\bf 142} (2020), 627-682, \url{https://muse.jhu.edu/issue/42062}.
\item Basu, R. and Ganguly, S. and Hegde, M. and Krishnapur, M.: Lower Deviations in β-ensembles and Law of Iterated Logarithm in Last Passage Percolation, to appear in \emph{Israel Journal of Mathematics}.
\item M. K. Ghosh and S. Pradhan: Zero-Sum Risk-Sensitive Stochastic Differential Games with Reflecting Diffusions in the Orthant, to appear in \emph{ESIAM: Control, Optimization and Calculus of Variations}, DOI Doi:10.105/cocv/2020029.
\item M. K. Ghosh, K. Suresh Kumar, C. Pal and S. Pradhan: Nonzero-sum risk-sensitive stochastic differential games with discounted costs, Stochastic Analysis and Applications, to appear in \emph{Stochastic Analysis and Applications}, DOI Doi: 10.1080/07362994.2020.1796707.
\item Iyer, S. K. and Yogeshwaran, D.: Thresholds for vanishing of isolated faces in random Cech and Vietoris-Rips Complexes, \emph{Annals de l'Institut Henri Poincar$\acute{e}$ - Probabiliti$\acute{e}$s et Statistiques} {\bf 56} (2020), 1869-1897, \url{https://doi.org/10.1214/19-AIHP1020}.
\item Datta, Basudeb and Gupta, Subhojoy: Semi-regular tilings of the hyperbolic plane, to appear in \emph{Discrete and Computational Geometry}, DOI https://doi.org/10.1007/s00454-019-00156-0.
\item Thirupathi Gudi, Ramesh Ch. Sau: Finite Element Analysis of the Constrained Dirichlet Boundary Control Problem Governed by the Diffusion Problem, to appear in \emph{ESAIM: Control, Optim. Calc. Var}, DOI doi: 10.1051/cocv/2019068..
\item Bhattacharyya, Tirthankar and Biswas, Anindya and Chandel, Vikramjeet Singh: On the Nevanlinna problem: characterization of all Schur–Agler class solutions affiliated with a given kernel, \emph{Studia Math.} {\bf 255} (2020), 83–107., DOI 10.4064/sm190505-8-10.
\item Bhattacharyya, Tirthankar and Das, Bata Krishna and Sau, Haripada: Toeplitz operators on the symmetrized bidisc, \emph{International Mathematics Research Notices (IMRN)} {\bf } (), DOI 10.1093/imrn/rnz333.
\item Pingali, Vamsi Pritham and Varolin, Dror: Nonuniformly flat affine algebraic hypersurfaces, to appear in \emph{Nagoya Mathematical Journal}, DOI https://doi.org/10.1017/nmj.2019.2, \url{https://www.cambridge.org/core/journals/nagoya-mathematical-journal/article/nonuniformly-flat-affine-algebraic-hypersurfaces/3704C1133CB22A05D7FB13765156FC4E}.
\item Alvarez-Consul, Luis and Garcia-Fernandez, Mario  and García-Prada, Oscar and Pingali, Vamsi Pritham: Gravitating vortices and the Einstein–Bogomol’nyi equations, to appear in \emph{Mathematische Annalen}, DOI https://doi.org/10.1007/s00208-020-01964-z, \url{https://link.springer.com/content/pdf/10.1007/s00208-020-01964-z.pdf}.
\item Datar, Ved and Jacob,Adam and Zhang, Yuguang: Adiabatic limits of anti-self-dual connections on collapsed K3 surfaces, to appear in \emph{Journal of differential geometry}, \url{https://arxiv.org/pdf/1809.08583.pdf}.
\item Datar, Ved and Pingali, Vamsi Pritham: On coupled constant scalar curvature Kaehler metrics, to appear in \emph{Journal of symplectic geometry}, \url{https://arxiv.org/pdf/1901.10454v2.pdf}.
\item Pradhan, P. and Sahu, B.: A characterization of the family of secant lines to a hyperbolic quadric in PG(3,q), q odd, \emph{Discrete Mathematics} {\bf 343} (2020), DOI 10.1016/j.disc.2020.112044, \url{https://doi.org/10.1016/j.disc.2020.112044}.
\item Arun Maiti: Quasi-vertex-transitive maps on the plane, \emph{Discrete Mathematics} {\bf 343} (2020), 111911-111917, \url{https://www.sciencedirect.com/science/article/pii/S0012365X20301035}.
\item Datt, Gopal and Kumar, Sanjay: A criterion for quasinormality in $\mathbb{C}^n$, \emph{Proc. Indian Acad. Sci. Math. Sci.} {\bf 129} (2019), Paper No. 11, 13 pp., DOI https://doi.org/10.1007/s12044-018-0449-5     (This article was published before the review period i.e., April 1, 2019, BUT this is not yet included on the page  https://siddhartha-gadgil.github.io/DeptWeb/pubs.html), \url{https://link.springer.com/article/10.1007/s12044-018-0449-5?shared-article-renderer}.
\item Hatui, Sumana: Schur multipliers of special p-groups of rank 2, \emph{Journal of Group Theory} {\bf 23} (2020), 85-95, DOI https://doi.org/10.1515/jgth-2019-0045, \url{https://www.degruyter.com/view/journals/jgth/23/1/article-p85.xml}.
\item Choudhury, Projesh Nath and Tsatsomeros Michael J.: Algorithmic Detection and Construction of N-matrices, \emph{Linear Algebra and Its Applications} {\bf 602} (2020), 46-56, DOI 10.1016/j.laa.2020.04.028, \url{https://doi.org/10.1016/j.laa.2020.04.028}.
\item Choudhury, Projesh Nath and Sivakumar, K.C.: Matrices with Positive Semidefinite Real Part, to appear in \emph{Linear and Multilinear Algebra}, DOI 10.1080/03081087.2019.1602588, \url{https://doi.org/10.1080/03081087.2019.1602588}.
\item Chim, Kwok Chi; Shorey, Tarlok Nath; Sinha, Sneh Bala: On Baker's explicit abc-conjecture, \emph{Publ. Math. Debrecen} {\bf 94} (2019), 435–453.
\item Carstensen, C. and Mallik, G. and Nataraj, N.: Nonconforming Finite Element Discretization for Semilinear Problems with Trilinear Nonlinearity, \emph{IMA Journal of Numerical Analysis} {\bf } (2020), DOI https://doi.org/10.1093/imanum/drz071, \url{https://doi.org/10.1093/imanum/drz071}.
\item Mal, Arpita and Sain, Debmalya and Paul, Kallol: On some geometric properties of operator spaces, \emph{Banach J. Math. Anal.} {\bf 13} (2019), 174-191, \url{https://projecteuclid.org/euclid.bjma/1543914019}.
\item Roth, Julien and Upadhyay, Abhitosh: On almost stable CMC hypersurfaces in manifolds of bounded sectional curvature, \emph{Bulletin of the Australian Mathematical Society} {\bf 101} (2020), 333-338, DOI https://doi.org/10.1017/S0004972719000935, \url{https://www.cambridge.org/core/journals/bulletin-of-the-australian-mathematical-society/article/on-almost-stable-cmc-hypersurfaces-in-manifolds-of-bounded-sectional-curvature/E0BF84D4D46C82F6D75CDCEAAF5DEE94}.
\item Roth, Julien and Upadhyay, Abhitosh: f-Biharmonic Submanifolds of Generalized Space Forms, \emph{Results in Mathematics} {\bf 75} (2020), DOI https://doi.org/10.1007/s00025-019-1142-4, \url{https://link.springer.com/article/10.1007/s00025-019-1142-4?shared-article-renderer#citeas}.
\item Barbosa, Ezequiel and Santana, Farley and Upadhyay, Abhitosh: On stable CMC free-boundary surfaces in a strictly convex domain of a bi-invariant Lie group, to appear in \emph{International Journal of Mathematics}, DOI https://doi.org/10.1142/S0129167X2050086X.
\end{enumerate}

\subsection{Papers in Conference proceedings}

\begin{enumerate}
\item A. Banerjee: Spectra as universal objects in categories of supports in {\em Banach Center Publications(Quantum Dynamics Proceedings)} Ludwik Dabrowski, Ryszard Nest, Adam Skalski (ed.) , Banach Center Publications , IMPAN, Warsaw, Poland (2020), (accepted).
\item IMPAN, Warsaw, Poland: 2020 in {\em (accepted)} No (ed.)   ().
\item S. Ghara and G. Misra: Decomposition of the tensor product of two Hilbert modules in {\em Operator Theory, Operator Algebras and Their Interactions with Geometry and Topology Ronald G. Douglas Memorial Volume} Curto, R.E., Helton, W., Lin, H., Tang, X., Yang, R., Yu, G. (ed.) , Operator Theory: Advances and Applications , Birkhauser (2020).
\item Birkhauser: 2020 in {\em } Yes (ed.) , Gadadhar Misra , Decomposition of the tensor product of two Hilbert modules (Operator Theory, Operator Algebras and Their Interactions with Geometry and Topology Ronald G. Douglas Memorial Volume), R. Curto, W. Helton, H. Lin, X. Tang, R. Yang, G. Yu.
\item Decomposition of the tensor product of two Hilbert modules: Operator Theory, Operator Algebras and Their Interactions with Geometry and Topology Ronald G. Douglas Memorial Volume in {\em R. Curto, W. Helton, H. Lin, X. Tang, R. Yang, G. Yu} Operator Theory: Advances and Applications (ed.) , Birkhauser , 2020 (), No.
\item 2020:  in {\em No}  (ed.)   ().
\end{enumerate}


\subsection{Books}



\subsection{Book Chapters}

\begin{enumerate}
\item Gupta, Subhojoy and Seshadri, Harish: Complex geometry of Teichmüller domains in {\em Handbook of Teichmüller theory} Papadopoulos, Athanase (ed.), IRMA Lect. Math. Theor. Phys, Eur. Math. Soc., Zürich (2020), 26.
\item Gupta, Subhojoy: Holomorphic quadratic differentials in Teichmüller theory in {\em Handbook of Teichmüller theory} Papadopoulos, Athanase (ed.), IRMA Lect. Math. Theor. Phys., Eur. Math. Soc., Zürich (2020), 36.
\end{enumerate}


\subsection{Preprints}

\begin{enumerate}
\item Khare, Apoorva: Sharp nonzero lower bounds for the Schur product theorem, submitted for publication, \url{https://arxiv.org/abs/1910.03537}.
\item Lübeck, Frank and Prasad, Dipendra and Ayyer, Arvind: A character relationship between symmetric group and hyperoctahedral group, preprint, \url{https://arxiv.org/abs/1912.08576}.
\item Ayyer, Arvind and Sinha, Shubham: The size of t-cores and hook lengths of random cells in random partitions, preprint, \url{https://arxiv.org/abs/1911.03135}.
\item Ayyer, Arvind: A simple symmetric exclusion process driven by an asymmetric tracer particle, preprint, \url{https://arxiv.org/abs/2001.02425}.
\item David Burns, Mahesh Kakde: On Weil-Etale Cohomology and Artin L-series, preprint, \url{https://nms.kcl.ac.uk/david.burns/web_page/bk.pdf}.
\item Gupta, Purvi and Wawrzyniak, Chloe Urbanski: STABILITY OF THE HULL(S) OF AN $n$-SPHERE IN $\mathbb{C}^n$, submitted for publication, \url{https://arxiv.org/pdf/2002.08699.pdf}.
\item Radhika Ganapathy: A Hecke algebra isomorphism over close local fields, preprint.
\item Miermont, Grégory and Sen, Sanchayan: On breadth-first constructions of scaling limits of random graphs and random unicellular maps, submitted for publication, \url{https://arxiv.org/abs/1908.04403}.
\item Gupta, Subhojoy and Mj, Mahan: Meromorphic projective structures, grafting and the monodromy map, preprint, \url{http://arxiv.org/abs/1904.03804}.
\item Gupta, Subhojoy: Monodromy groups of $CP^1$-structures on punctured surfaces, preprint, \url{https://arxiv.org/abs/1909.10771}.
\item Biswas, Indranil and Gupta, Subhojoy and Mj, Mahan and Whang, Junho Peter: Surface group representations in SL(2,C) with finite mapping class orbits, preprint, \url{https://arxiv.org/abs/1707.00071}.
\item Ganguly, Pritam and Thangavelu,Sundaram: On lacunary spherical maximal function on Heisenberg groups, preprint.
\item Bhattacharyya, Tirthankar and Kumar, Poornendu and Sau, Haripada: Distinguished Varieties Through the Berger--Coburn--Lebow Theorem, submitted for publication, \url{https://arxiv.org/pdf/2001.01410.pdf}.
\item Bhattacharyya, T. and  Biswas, A. and Maitra, A.: An unbounded realization of the symmetrized bidisc, submitted for publication, \url{https://arxiv.org/pdf/2005.00289.pdf}.
\item Pingali, Vamsi Pritham: The deformed Hermitian Yang-Mills equation on three-folds, preprint, \url{https://arxiv.org/abs/1910.01870}.
\item Garcia-Fernandez, Mario and Pingali, Vamsi Pritham and Yao, Chengjian: Gravitating vortices with positive curvature, preprint, \url{https://arxiv.org/abs/1911.09616}.
\item Quddus, Safdar: Equivariant Cartan Homotopy Formula for $DG$-algebra., preprint, \url{https://arxiv.org/abs/2002.02192}.
\item Arun Maiti: OPTIMAL INDEX GROWTH OF CLOSED GEODESICS, submitted for publication.
\item Poria, Anirudha: Uncertainty Principles for the Fourier and the Short-Time Fourier Transforms, submitted for publication, \url{https://arxiv.org/abs/2004.04184}.
\item Poria, Anirudha: Uncertainty principles for the Opdam-Cherednik transform on modulation spaces, submitted for publication, \url{https://arxiv.org/abs/2005.14274}.
\item Bera, Sudip: ENUMERATION OF WEIGHTED PATHS ON A DIGRAPH AND BLOCK HOOK DETERMINANT, submitted for publication.
\item Behera, Kiran Kumar: Self-inversive polynomials from rational transformation of linear functionals, submitted for publication, \url{https://arxiv.org/abs/1909.12548}.
\item Hatui, Sumana and Singla Pooja: On Schur multiplier and Projective representations of Heisenberg groups, submitted for publication, \url{https://arxiv.org/abs/1909.06589}.
\item Chattopahyay, Arup and Sarkar, Jaydeb and Sarkar, Srijan: Multiplicities, invariant subspaces and an additive formula., preprint, \url{https://arxiv.org/abs/1812.05435}.
\item Ghara, Soumitra and Kumar, Surjit and Pramanick, Paramita: $K$-homogeneous tuple of operators on bounded symmetric domains, preprint.
\item Toft, J. and Gumber, A. and Manna, R. and Ratnakumar, P. K.: Translation and modulation invariant Hilbert spaces, submitted for publication, \url{https://arxiv.org/abs/2004.02430v1}.
\end{enumerate}

% from web page

\subsection{Publications (all kinds) from web page}

\begin{enumerate}
\item Aas, Erik and Ayyer, Arvind and Linusson, Svante and Potka,
Samu: The exact phase diagram for a semipermeable {TASEP} with
nonlocal boundary jumps, \emph{J. Phys. A} {\bf 52} (2019), 355001, 19, DOI 10.1088/1751-8121/ab2e96, \url{https://doi.org/10.1088/1751-8121/ab2e96}.
\item Aiyappan, S. and Jose, Editha C. and Lomerio, Ivy Carol B. and
Nandakumaran, A. K.: Control problem on a rough circular domain and homogenization, \emph{Asymptot. Anal.} {\bf 115} (2019), 19--46, DOI 10.3233/asy-191526, \url{https://doi.org/10.3233/asy-191526}.
\item Aiyappan, S. and Nandakumaran, A. K. and Prakash, Ravi: Locally periodic unfolding operator for highly oscillating
rough domains, \emph{Ann. Mat. Pura Appl. (4)} {\bf 198} (2019), 1931--1954, DOI 10.1007/s10231-019-00848-7, \url{https://doi.org/10.1007/s10231-019-00848-7}.
\item Aiyappan, S. and Nandakumaran, A. K. and Prakash, Ravi: Generalization of unfolding operator for highly oscillating
smooth boundary domains and homogenization, \emph{Calc. Var. Partial Differential Equations} {\bf 57} (2018), Art. 86, 30, DOI 10.1007/s00526-018-1354-6, \url{https://doi.org/10.1007/s00526-018-1354-6}.
\item Aiyappan, S. and Nandakumaran, A. K. and Sufian, Abu: Asymptotic analysis of a boundary optimal control problem on a
general branched structure, \emph{Math. Methods Appl. Sci.} {\bf 42} (2019), 6407--6434, DOI 10.1002/mma.5748, \url{https://doi.org/10.1002/mma.5748}.
\item Aiyappan, S. and Sardar, Bidhan Chandra: Biharmonic equation in a highly oscillating domain and
homogenization of an associated control problem, \emph{Appl. Anal.} {\bf 98} (2019), 2783--2801, DOI 10.1080/00036811.2018.1471207, \url{https://doi.org/10.1080/00036811.2018.1471207}.
\item Akutagawa, Kazuo and Endo, Hisaaki and Seshadri, Harish: A gap theorem for positive {E}instein metrics on the
four-sphere, \emph{Math. Ann.} {\bf 373} (2019), 1329--1339, DOI 10.1007/s00208-018-1749-x, \url{https://doi.org/10.1007/s00208-018-1749-x}.
\item Anamby, Pramath and Das, Soumya: Distinguishing {H}ermitian cusp forms of degree 2 by a certain
subset of all {F}ourier coefficients, \emph{Publ. Mat.} {\bf 63} (2019), 307--341, DOI 10.5565/PUBLMAT6311911, \url{https://doi.org/10.5565/PUBLMAT6311911}.
\item Anamby, Pramath and Das, Soumya: Bounds for the {P}etersson norms of the pullbacks of
{S}aito-{K}urokawa lifts, \emph{J. Number Theory} {\bf 191} (2018), 289--304, DOI 10.1016/j.jnt.2018.03.011, \url{https://doi.org/10.1016/j.jnt.2018.03.011}.
\item Arunkumar, G. and Kus, Deniz and Venkatesh, R.: Root multiplicities for {B}orcherds algebras and graph
coloring, \emph{J. Algebra} {\bf 499} (2018), 538--569, DOI 10.1016/j.jalgebra.2017.11.050, \url{https://doi.org/10.1016/j.jalgebra.2017.11.050}.
\item Ayyer, Arvind and Behrend, Roger E.: Factorization theorems for classical group characters, with
applications to alternating sign matrices and plane
partitions, \emph{J. Combin. Theory Ser. A} {\bf 165} (2019), 78--105, DOI 10.1016/j.jcta.2019.01.001, \url{https://doi.org/10.1016/j.jcta.2019.01.001}.
\item Ayyer, Arvind and Bouttier, J\'{e}r\'{e}mie and Corteel, Sylvie and
Linusson, Svante and Nunzi, Fran\c{c}ois: Bumping sequences and multispecies juggling, \emph{Adv. in Appl. Math.} {\bf 98} (2018), 100--126, DOI 10.1016/j.aam.2018.03.001, \url{https://doi.org/10.1016/j.aam.2018.03.001}.
\item Ayyer, Arvind and Finn, Caley and Roy, Dipankar: Matrix product solution of a left-permeable two-species
asymmetric exclusion process, \emph{Phys. Rev. E} {\bf 97} (2018), 012151, 10, DOI 10.1103/physreve.97.012151, \url{https://doi.org/10.1103/physreve.97.012151}.
\item Ayyer, Arvind and Finn, Caley and Roy, Dipankar: The phase diagram for a multispecies left-permeable asymmetric
exclusion process, \emph{J. Stat. Phys.} {\bf 174} (2019), 605--621, DOI 10.1007/s10955-018-2183-x, \url{https://doi.org/10.1007/s10955-018-2183-x}.
\item Ayyer, Arvind and Linusson, Svante: Reverse juggling processes, \emph{Random Structures Algorithms} {\bf 55} (2019), 56--72, DOI 10.1002/rsa.20825, \url{https://doi.org/10.1002/rsa.20825}.
\item Ayyer, Arvind and Prasad, Amritanshu and Spallone, Steven: Odd partitions in {Y}oung's lattice, \emph{S\'{e}m. Lothar. Combin.} {\bf 75} (2018), Art. B75g, 13.
\item Ayyer, Arvind and Ramassamy, Sanjay: The {H}ilbert-{G}alton board, \emph{ALEA Lat. Am. J. Probab. Math. Stat.} {\bf 15} (2018), 755--774, DOI 10.30757/alea.v15-28, \url{https://doi.org/10.30757/alea.v15-28}.
\item B\"{o}cherer, Siegfried and Das, Soumya: Cuspidality and the growth of {F}ourier coefficients of
modular forms, \emph{J. Reine Angew. Math.} {\bf 741} (2018), 161--178, DOI 10.1515/crelle-2015-0075, \url{https://doi.org/10.1515/crelle-2015-0075}.
\item B\"{o}cherer, Siegfried and Das, Soumya: Petersson norms of not necessarily cuspidal {J}acobi modular
forms and applications, \emph{Adv. Math.} {\bf 336} (2018), 335--376, DOI 10.1016/j.aim.2018.07.033, \url{https://doi.org/10.1016/j.aim.2018.07.033}.
\item Bagchi, Sayan and Thangavelu, Sundaram: Weighted norm inequalities for {W}eyl multipliers and fourier
multipliers on the {H}eisenberg group, \emph{J. Anal. Math.} {\bf 136} (2018), 1--29, DOI 10.1007/s11854-018-0053-8, \url{https://doi.org/10.1007/s11854-018-0053-8}.
\item Baklouti, Ali and Thangavelu, Sundaram: Hardy and {M}iyachi theorems for {H}eisenberg motion groups, \emph{Nagoya Math. J.} {\bf 229} (2018), 1--20, DOI 10.1017/nmj.2016.58, \url{https://doi.org/10.1017/nmj.2016.58}.
\item Balakumar, G. P. and Borah, Diganta and Mahajan, Prachi and
Verma, Kaushal: Remarks on the higher dimensional {S}uita conjecture, \emph{Proc. Amer. Math. Soc.} {\bf 147} (2019), 3401--3411, DOI 10.1090/proc/14421, \url{https://doi.org/10.1090/proc/14421}.
\item Balhara, Rakesh: Hardy's inequality for the fractional powers of the {G}rushin
operator, \emph{Proc. Indian Acad. Sci. Math. Sci.} {\bf 129} (2019), Art. 33, 25, DOI 10.1007/s12044-019-0471-2, \url{https://doi.org/10.1007/s12044-019-0471-2}.
\item Balodi, Mamta and Palcoux, Sebastien: On {B}oolean intervals of finite groups, \emph{J. Combin. Theory Ser. A} {\bf 157} (2018), 49--69, DOI 10.1016/j.jcta.2018.02.004, \url{https://doi.org/10.1016/j.jcta.2018.02.004}.
\item Banerjee, Abhishek: On {A}uslander's formula and cohereditary torsion pairs, \emph{Commun. Contemp. Math.} {\bf 20} (2018), 1750071, 27, DOI 10.1142/S0219199717500717, \url{https://doi.org/10.1142/S0219199717500717}.
\item Banerjee, Abhishek: Modular {H}ecke algebras over {M}\"{o}bius categories, \emph{J. Geom. Phys.} {\bf 131} (2018), 23--40, DOI 10.1016/j.geomphys.2018.04.008, \url{https://doi.org/10.1016/j.geomphys.2018.04.008}.
\item Banerjee, Abhishek: On differential torsion theories and rings with several
objects, \emph{Canad. Math. Bull.} {\bf 62} (2019), 703--714, DOI 10.4153/s0008439518000656, \url{https://doi.org/10.4153/s0008439518000656}.
\item Banerjee, Abhishek: A topological {N}ullstellensatz for tensor-triangulated
categories, \emph{C. R. Math. Acad. Sci. Paris} {\bf 356} (2018), 365--375, DOI 10.1016/j.crma.2018.02.012, \url{https://doi.org/10.1016/j.crma.2018.02.012}.
\item Banerjee, Abhishek: Quasimodular {H}ecke algebras and {H}opf actions, \emph{J. Noncommut. Geom.} {\bf 12} (2018), 1041--1080, DOI 10.4171/JNCG/297, \url{https://doi.org/10.4171/JNCG/297}.
\item Banerjee, Abhishek: Completions of monoid objects and descent results, \emph{J. Algebra} {\bf 507} (2018), 362--399, DOI 10.1016/j.jalgebra.2018.04.022, \url{https://doi.org/10.1016/j.jalgebra.2018.04.022}.
\item Banerjee, Abhishek and Kour, Surjeet: {$(A,\delta)$}-modules, {H}ochschild homology and higher
derivations, \emph{Ann. Mat. Pura Appl. (4)} {\bf 198} (2019), 1781--1802, DOI 10.1007/s10231-019-00844-x, \url{https://doi.org/10.1007/s10231-019-00844-x}.
\item Basu, Arnab and Ghosh, Mrinal K.: Nonzero-sum risk-sensitive stochastic games on a countable
state space, \emph{Math. Oper. Res.} {\bf 43} (2018), 516--532, DOI 10.1287/moor.2017.0870, \url{https://doi.org/10.1287/moor.2017.0870}.
\item Belton, Alexander and Guillot, Dominique and Khare, Apoorva
and Putinar, Mihai: Simultaneous kernels of matrix {H}adamard powers, \emph{Linear Algebra Appl.} {\bf 576} (2019), 142--157, DOI 10.1016/j.laa.2018.03.035, \url{https://doi.org/10.1016/j.laa.2018.03.035}.
\item Benedetti, Carolina and Gonz\'{a}lez D'Le\'{o}n, Rafael S. and Hanusa,
Christopher R. H. and Harris, Pamela E. and Khare, Apoorva and
Morales, Alejandro H. and Yip, Martha: The volume of the caracol polytope, \emph{S\'{e}m. Lothar. Combin.} {\bf 80B} (2018), Art. 87, 12.
\item Benedetti, Carolina and Gonz\'{a}lez D'Le\'{o}n, Rafael S. and Hanusa,
Christopher R. H. and Harris, Pamela E. and Khare, Apoorva and
Morales, Alejandro H. and Yip, Martha: A combinatorial model for computing volumes of flow polytopes, \emph{Trans. Amer. Math. Soc.} {\bf 372} (2019), 3369--3404, DOI 10.1090/tran/7743, \url{https://doi.org/10.1090/tran/7743}.
\item Bera, Sayani and Verma, Kaushal: Some aspects of shift-like automorphisms of {$\Bbb C^k$}, \emph{Proc. Indian Acad. Sci. Math. Sci.} {\bf 128} (2018), Art. 10, 48, DOI 10.1007/s12044-018-0388-1, \url{https://doi.org/10.1007/s12044-018-0388-1}.
\item Bharali, Gautam and Biswas, Indranil and Divakaran, Divakaran
and Janardhanan, Jaikrishnan: Proper holomorphic mappings onto symmetric products of a
{R}iemann surface, \emph{Doc. Math.} {\bf 23} (2018), 1291--1311.
\item Bharali, Gautam and Chandel, Vikramjeet Singh: Pick interpolation on the polydisc: small families of
sufficient kernels, \emph{Complex Anal. Oper. Theory} {\bf 13} (2019), 2069--2093, DOI 10.1007/s11785-017-0701-5, \url{https://doi.org/10.1007/s11785-017-0701-5}.
\item Bhattacharyya, Tirthankar and Rao, T. S. S. R. K.: Survey of last ten years of work done in {I}ndia in some
selected areas of functional analysis and operator theory, \emph{Indian J. Pure Appl. Math.} {\bf 50} (2019), 599--617, DOI 10.1007/s13226-019-0345-4, \url{https://doi.org/10.1007/s13226-019-0345-4}.
\item Bhattacharyya, Tirthankar and Sau, Haripada: Holomorphic functions on the symmetrized bidisk---realization,
interpolation and extension, \emph{J. Funct. Anal.} {\bf 274} (2018), 504--524, DOI 10.1016/j.jfa.2017.09.013, \url{https://doi.org/10.1016/j.jfa.2017.09.013}.
\item Bhimani, Divyang G. and Balhara, Rakesh and Thangavelu,
Sundaram: Hermite multipliers on modulation spaces, \emph{Analysis and partial differential equations: perspectives from
developing countries} {\bf 275} (2019), 42--64.
\item Bhosle, Usha N.: Hitchin pairs on reducible curves, \emph{Internat. J. Math.} {\bf 29} (2018), 1850015, 49, DOI 10.1142/S0129167X18500155, \url{https://doi.org/10.1142/S0129167X18500155}.
\item Biswas, Indranil and D'Mello, Shane and Mukherjee, Ritwik and
Pingali, Vamsi P.: Rational cuspidal curves on del-{P}ezzo surfaces, \emph{J. Singul.} {\bf 17} (2018), 91--107.
\item Biswas, Indranil and Dumitrescu, Sorin and Gupta, Subhojoy: Branched projective structures on a {R}iemann surface and
logarithmic connections, \emph{Doc. Math.} {\bf 24} (2019), 2299--2337.
\item Biswas, Indranil and Pingali, Vamsi Pritham: A characterization of finite vector bundles on {G}auduchon
astheno-{K}\"{a}hler manifolds, \emph{\'{E}pijournal Geom. Alg\'{e}brique} {\bf 2} (2018), Art. 6, 13.
\item Biswas, Rahul and Dond, Asha K. and Gudi, Thirupathi: Edge patch-wise local projection stabilized nonconforming
{FEM} for the {O}seen problem, \emph{Comput. Methods Appl. Math.} {\bf 19} (2019), 189--214, DOI 10.1515/cmam-2018-0020, \url{https://doi.org/10.1515/cmam-2018-0020}.
\item Biswas, Shibananda and Ghosh, Gargi and Misra, Gadadhar and
Shyam Roy, Subrata: On reducing submodules of {H}ilbert modules with
{${S}_n$}-invariant kernels, \emph{J. Funct. Anal.} {\bf 276} (2019), 751--784, DOI 10.1016/j.jfa.2018.10.025, \url{https://doi.org/10.1016/j.jfa.2018.10.025}.
\item Boggarapu, Pradeep and Roncal, Luz and Thangavelu, Sundaram: On extension problem, trace {H}ardy and {H}ardy's inequalities
for some fractional {L}aplacians, \emph{Commun. Pure Appl. Anal.} {\bf 18} (2019), 2575--2605, DOI 10.3934/cpaa.2019116, \url{https://doi.org/10.3934/cpaa.2019116}.
\item Borah, Diganta and Haridas, Pranav and Verma, Kaushal: Comments on the {G}reen's function of a planar domain, \emph{Anal. Math. Phys.} {\bf 8} (2018), 383--414, DOI 10.1007/s13324-017-0177-5, \url{https://doi.org/10.1007/s13324-017-0177-5}.
\item Brenner, Susanne C. and Gudi, Thirupathi and Porwal, Kamana
and Sung, Li-Yeng: A {M}orley finite element method for an elliptic distributed
optimal control problem with pointwise state and control
constraints, \emph{ESAIM Control Optim. Calc. Var.} {\bf 24} (2018), 1181--1206, DOI 10.1051/cocv/2017031, \url{https://doi.org/10.1051/cocv/2017031}.
\item Burton, Benjamin A. and Datta, Basudeb and Singh, Nitin and
Spreer, Jonathan: A construction principle for tight and minimal triangulations
of manifolds, \emph{Exp. Math.} {\bf 27} (2018), 22--36, DOI 10.1080/10586458.2016.1212747, \url{https://doi.org/10.1080/10586458.2016.1212747}.
\item Carstensen, Carsten and Dond, Asha K. and Rabus, Hella: Quasi-optimality of adaptive mixed {FEM}s for non-selfadjoint
indefinite second-order linear elliptic problems, \emph{Comput. Methods Appl. Math.} {\bf 19} (2019), 233--250, DOI 10.1515/cmam-2019-0034, \url{https://doi.org/10.1515/cmam-2019-0034}.
\item Chandel, Vikramjeet Singh: The three-point {P}ick-{N}evanlinna interpolation problem on
the polydisc, \emph{Complex Var. Elliptic Equ.} {\bf 63} (2018), 1341--1352, DOI 10.1080/17476933.2017.1370461, \url{https://doi.org/10.1080/17476933.2017.1370461}.
\item Chatterjee, Esha and Hassan, Sk. Sarif: On the asymptotic character of a generalized rational
difference equation, \emph{Discrete Contin. Dyn. Syst.} {\bf 38} (2018), 1707--1718, DOI 10.3934/dcds.2018070, \url{https://doi.org/10.3934/dcds.2018070}.
\item Choudhury, Projesh Nath and Kannan, M. Rajesh and Sivakumar,
K. C.: A note on linear preservers of semipositive and minimally
semipositive matrices, \emph{Electron. J. Linear Algebra} {\bf 34} (2018), 687--694, DOI 10.13001/1081-3810.3864, \url{https://doi.org/10.13001/1081-3810.3864}.
\item Ciaurri, \'{O}scar and Roncal, Luz and Thangavelu, Sundaram: Hardy-type inequalities for fractional powers of the
{D}unkl-{H}ermite operator, \emph{Proc. Edinb. Math. Soc. (2)} {\bf 61} (2018), 513--544, DOI 10.1017/s0013091517000311, \url{https://doi.org/10.1017/s0013091517000311}.
\item Das, Soumya and Jha, Abhash Kumar: Analytic properties of twisted real-analytic {H}ermitian
{K}lingen type {E}isenstein series and applications, \emph{Abh. Math. Semin. Univ. Hambg.} {\bf 89} (2019), 105--116, DOI 10.1007/s12188-019-00206-7, \url{https://doi.org/10.1007/s12188-019-00206-7}.
\item Das, Soumya and Khan, Rizwanur: The third moment of symmetric square {$L$}-functions, \emph{Q. J. Math.} {\bf 69} (2018), 1063--1087, DOI 10.1093/qmath/hay012, \url{https://doi.org/10.1093/qmath/hay012}.
\item Das, Soumya and Kohnen, Winfried: On sign changes of eigenvalues of {S}iegel cusp forms of genus
2 in prime powers, \emph{Acta Arith.} {\bf 183} (2018), 167--172, DOI 10.4064/aa170419-4-11, \url{https://doi.org/10.4064/aa170419-4-11}.
\item Das, Soumya and Pal, Ritwik: The first negative eigenvalue of {Y}oshida lifts, \emph{Res. Number Theory} {\bf 5} (2019), Paper No. 20, 9, DOI 10.1007/s40993-019-0158-x, \url{https://doi.org/10.1007/s40993-019-0158-x}.
\item Datta, Basudeb and Maity, Dipendu: Semi-equivelar maps on the torus and the {K}lein bottle are
{A}rchimedean, \emph{Discrete Math.} {\bf 341} (2018), 3296--3309, DOI 10.1016/j.disc.2018.08.016, \url{https://doi.org/10.1016/j.disc.2018.08.016}.
\item Dond, Asha K. and Gudi, Thirupathi: Patch-wise local projection stabilized finite element methods
for convection-diffusion-reaction problems, \emph{Numer. Methods Partial Differential Equations} {\bf 35} (2019), 638--663, DOI 10.1002/num.22317, \url{https://doi.org/10.1002/num.22317}.
\item Dond, Asha K. and Gudi, Thirupathi and Sau, Ramesh CH.: An error analysis of discontinuous finite element methods for
the optimal control problems governed by {S}tokes equation, \emph{Numer. Funct. Anal. Optim.} {\bf 40} (2019), 421--460, DOI 10.1080/01630563.2018.1538158, \url{https://doi.org/10.1080/01630563.2018.1538158}.
\item Fritz, Tobias and Gadgil, Siddhartha and Khare, Apoorva and Nielsen, Pace P. and Silberman, Lior and Tao, Terence: Homogeneous length functions on groups, \emph{Algebra Number Theory} {\bf 12} (2018), 1773--1786, DOI 10.2140/ant.2018.12.1773, \url{https://doi.org/10.2140/ant.2018.12.1773}.
\item Gaddam, Sharat and Gudi, Thirupathi: Inhomogeneous {D}irichlet boundary condition in the {\it a
posteriori} error control of the obstacle problem, \emph{Comput. Math. Appl.} {\bf 75} (2018), 2311--2327, DOI 10.1016/j.camwa.2017.12.010, \url{https://doi.org/10.1016/j.camwa.2017.12.010}.
\item Gaddam, Sharat and Gudi, Thirupathi: Bubbles enriched quadratic finite element method for the
3{D}-elliptic obstacle problem, \emph{Comput. Methods Appl. Math.} {\bf 18} (2018), 223--236, DOI 10.1515/cmam-2017-0018, \url{https://doi.org/10.1515/cmam-2017-0018}.
\item Garani, Shayan Srinivasa and Seshadri, Harish: An algorithmic approach to {S}outh {I}ndian classical music, \emph{J. Math. Music} {\bf 13} (2019), 107--134, DOI 10.1080/17459737.2019.1604845, \url{https://doi.org/10.1080/17459737.2019.1604845}.
\item Gaussier, Herv\'{e} and Seshadri, Harish: Holomorphic motions and complex geometry, \emph{Proc. Amer. Math. Soc.} {\bf 147} (2019), 301--313, DOI 10.1090/proc/14217, \url{https://doi.org/10.1090/proc/14217}.
\item Gaussier, Herv\'{e} and Seshadri, Harish: On the {G}romov hyperbolicity of convex domains in {$\Bbb
C^n$}, \emph{Comput. Methods Funct. Theory} {\bf 18} (2018), 617--641, DOI 10.1007/s40315-018-0243-5, \url{https://doi.org/10.1007/s40315-018-0243-5}.
\item Ghara, Soumitra and Kumar, Surjit: On the sum of two subnormal kernels, \emph{J. Math. Anal. Appl.} {\bf 469} (2019), 1015--1027, DOI 10.1016/j.jmaa.2018.09.048, \url{https://doi.org/10.1016/j.jmaa.2018.09.048}.
\item Ghosh, Mrinal K. and Pradhan, Somnath: Risk-sensitive stochastic differential games with reflecting
diffusions, \emph{Stoch. Anal. Appl.} {\bf 36} (2018), 1--27, DOI 10.1080/07362994.2017.1356732, \url{https://doi.org/10.1080/07362994.2017.1356732}.
\item Gudi, Thirupathi and Majumder, Papri: Convergence analysis of finite element method for a parabolic
obstacle problem, \emph{J. Comput. Appl. Math.} {\bf 357} (2019), 85--102, DOI 10.1016/j.cam.2019.02.026, \url{https://doi.org/10.1016/j.cam.2019.02.026}.
\item Gudi, Thirupathi and Majumder, Papri: Conforming and discontinuous {G}alerkin {FEM} in space for
solving parabolic obstacle problem, \emph{Comput. Math. Appl.} {\bf 78} (2019), 3896--3915, DOI 10.1016/j.camwa.2019.06.022, \url{https://doi.org/10.1016/j.camwa.2019.06.022}.
\item Gupta, Rajeev and Kumar, Surjit and Trivedi, Shailesh: Von {N}eumann's inequality for commuting operator-valued
multishifts, \emph{Proc. Amer. Math. Soc.} {\bf 147} (2019), 2599--2608, DOI 10.1090/proc/14410, \url{https://doi.org/10.1090/proc/14410}.
\item Gupta, Rajeev and Ramiz Reza, Md.: Operator space structures on {$\ell^1(n)$}, \emph{Houston J. Math.} {\bf 44} (2018), 1205--1212.
\item Gupta, S. C.: The classical {S}tefan problem, \emph{} {\bf } (2018), xxiii+726.
\item Gupta, Subhojoy: Limits of harmonic maps and crowned hyperbolic surfaces, \emph{Trans. Amer. Math. Soc.} {\bf 372} (2019), 7573--7596, DOI 10.1090/tran/7777, \url{https://doi.org/10.1090/tran/7777}.
\item Gupta, Subhojoy and Wolf, Michael: Meromorphic quadratic differentials and measured foliations on
a {R}iemann surface, \emph{Math. Ann.} {\bf 373} (2019), 73--118, DOI 10.1007/s00208-018-1674-z, \url{https://doi.org/10.1007/s00208-018-1674-z}.
\item Hazra, Somnath: Homogeneous 2-shifts, \emph{Complex Anal. Oper. Theory} {\bf 13} (2019), 1729--1763, DOI 10.1007/s11785-018-0822-5, \url{https://doi.org/10.1007/s11785-018-0822-5}.
\item Iyer, Srikanth K.: The random connection model: connectivity, edge lengths, and
degree distributions, \emph{Random Structures Algorithms} {\bf 52} (2018), 283--300, DOI 10.1002/rsa.20741, \url{https://doi.org/10.1002/rsa.20741}.
\item Jha, Abhash Kumar and Juyal, Abhishek and Pandey, Manish
Kumar: On simultaneous non-vanishing of twisted {$L$}-functions
associated to newforms on {$\Gamma_0(N)$}, \emph{J. Ramanujan Math. Soc.} {\bf 34} (2019), 245--252.
\item Jha, Abhash Kumar and Sahu, Brundaban: Rankin-{C}ohen brackets on {J}acobi forms of several variables
and special values of certain {D}irichlet series, \emph{Int. J. Number Theory} {\bf 15} (2019), 925--933, DOI 10.1142/S1793042119500490, \url{https://doi.org/10.1142/S1793042119500490}.
\item Jha, Abhash Kumar and Sahu, Brundaban: Differential operators on {J}acobi forms and special values of
certain {D}irichlet series, \emph{Automorphic forms and related topics} {\bf 732} (2019), 91--99, DOI 10.1090/conm/732/14793, \url{https://doi.org/10.1090/conm/732/14793}.
\item Kanukurthi, Bhavana and Bhavana Obbattu, Sai Lakshmi and
Sekar, Sruthi: Non-malleable randomness encoders and their applications, \emph{Advances in cryptology---{EUROCRYPT} 2018. {P}art {III}} {\bf 10822} (2018), 589--617.
\item Kashyap, G. A. R. S. R. K. and Bapat, D. and Das, D. and Gowaikar, R. D. and  Amritkar, R. E. and Rangarajan, G. and V. Ravindranath and G. Ambika: Synapse loss and progress of Alzheimers disease - A network model, \emph{Scientific Reports} {\bf 9} (2019).
\item Kataria, Kuldeep Kumar and Vellaisamy, Palaniappan: On the convolution of {M}ittag-{L}effler distributions and its
applications to fractional point processes, \emph{Stoch. Anal. Appl.} {\bf 37} (2019), 115--122, DOI 10.1080/07362994.2018.1538803, \url{https://doi.org/10.1080/07362994.2018.1538803}.
\item Khan, Mohammad Yasir Feroz and Biswas, Rahul and Sau, Ramesh
Chandra: Instability of {L}agrange interpolation on {$H^1(\Omega)\cap
C(\overline\Omega)$}, \emph{Math. Student} {\bf 88} (2019), 153--158.
\item Khare, Apoorva: Generalized nil-{C}oxeter algebras over discrete complex
reflection groups, \emph{Trans. Amer. Math. Soc.} {\bf 370} (2018), 2971--2999, DOI 10.1090/tran/7304, \url{https://doi.org/10.1090/tran/7304}.
\item Khare, Apoorva and Tikaradze, Akaki: A {C}arlitz--von {S}taudt type theorem for finite rings, \emph{Linear Algebra Appl.} {\bf 568} (2019), 106--126, DOI 10.1016/j.laa.2018.05.035, \url{https://doi.org/10.1016/j.laa.2018.05.035}.
\item Krakovski, Roi and Onn, Uri and Singla, Pooja: Regular characters of groups of type {$\ssf A_n$} over
discrete valuation rings, \emph{J. Algebra} {\bf 496} (2018), 116--137, DOI 10.1016/j.jalgebra.2017.10.018, \url{https://doi.org/10.1016/j.jalgebra.2017.10.018}.
\item Lin, X. and Kulkarni, P. and Bocci, F. and Schafer, N. and Roy, S. and Tsai, M-Y. and He, Y. and Chen, Y and Rajagopalan, K. and Mooney, S. and Zeng, Y. and Weninger, K. and Grishaev, A. and Onuchic, J. and Levine, H. and Wolynes, P. and Salgia, R. and Rangarajan, G. and Uversky, V. and Orban, J. and Jolly, M. K.: Structural and Dynamical Order of a Disordered Protein: Molecular Insights into Conformational Switching of PAGE4 at the Systems Level, \emph{Biomolecules} {\bf 9} (2019), 77.
\item Mahajan, Prachi and Verma, Kaushal: A comparison of two biholomorphic invariants, \emph{Internat. J. Math.} {\bf 30} (2019), 1950012, 16, DOI 10.1142/S0129167X19500125, \url{https://doi.org/10.1142/S0129167X19500125}.
\item Maity, Dipendu: Pentagonal maps on the torus and the plane, \emph{Beitr. Algebra Geom.} {\bf 60} (2019), 17--37, DOI 10.1007/s13366-018-0405-7, \url{https://doi.org/10.1007/s13366-018-0405-7}.
\item Maity, Dipendu and Upadhyay, Ashish Kumar: On centrally symmetric manifolds, \emph{J. Ramanujan Math. Soc.} {\bf 34} (2019), 21--27.
\item Mal, Arpita and Paul, Kallol and Rao, T. S. S. R. K. and Sain,
Debmalya: Approximate {B}irkhoff-{J}ames orthogonality and smoothness in
the space of bounded linear operators, \emph{Monatsh. Math.} {\bf 190} (2019), 549--558, DOI 10.1007/s00605-019-01289-3, \url{https://doi.org/10.1007/s00605-019-01289-3}.
\item Mal, Arpita and Sain, Debmalya and Paul, Kallol: On some geometric properties of operator spaces, \emph{Banach J. Math. Anal.} {\bf 13} (2019), 174--191, DOI 10.1215/17358787-2018-0021, \url{https://doi.org/10.1215/17358787-2018-0021}.
\item Manfio, F. and Turgay, N. C. and Upadhyay, A.: Biconservative submanifolds in {$\Bbb{S}^n\times\Bbb{R}$} and
{$\Bbb{H}^n\times\Bbb{R}$}, \emph{J. Geom. Anal.} {\bf 29} (2019), 283--298, DOI 10.1007/s12220-018-9990-9, \url{https://doi.org/10.1007/s12220-018-9990-9}.
\item Manikandan, K. and Vishnu Priya, N. and Senthilvelan, M. and
Sankaranarayanan, R.: Deformation of dark solitons in a {$\Cal{PT}$}-invariant
variable coefficients nonlocal nonlinear {S}chr\"{o}dinger
equation, \emph{Chaos} {\bf 28} (2018), 083103, 12, DOI 10.1063/1.5039901, \url{https://doi.org/10.1063/1.5039901}.
\item Misra, Gadadhar: Operators in the {C}owen-{D}ouglas class and related topics, \emph{Handbook of analytic operator theory} {\bf } (2019), 87--137.
\item Misra, Gadadhar and Pal, Avijit: Curvature inequalities for operators in the {C}owen-{D}ouglas
class and localization of the {W}allach set, \emph{J. Anal. Math.} {\bf 136} (2018), 31--54, DOI 10.1007/s11854-018-0054-7, \url{https://doi.org/10.1007/s11854-018-0054-7}.
\item Misra, Gadadhar and Pal, Avijit and Varughese, Cherian: Contractivity and complete contractivity for finite
dimensional {B}anach spaces, \emph{J. Operator Theory} {\bf 82} (2019), 23--47, DOI 10.7900/jot.2018jun13.2225, \url{https://doi.org/10.7900/jot.2018jun13.2225}.
\item Misra, Gadadhar and Reza, Md. Ramiz: Curvature inequalities and extremal operators, \emph{Illinois J. Math.} {\bf 63} (2019), 193--217, DOI 10.1215/00192082-7768711, \url{https://doi.org/10.1215/00192082-7768711}.
\item Mukherjee, Sajal Kumar and Bera, Sudip: Combinatorial proofs of the {N}ewton-{G}irard and
{C}hapman-{C}ostas-{S}antos identities, \emph{Discrete Math.} {\bf 342} (2019), 1577--1580, DOI 10.1016/j.disc.2019.02.013, \url{https://doi.org/10.1016/j.disc.2019.02.013}.
\item Narayanan, E. K. and Singla, Pooja: On monomial representations of finitely generated nilpotent
groups, \emph{Comm. Algebra} {\bf 46} (2018), 2319--2331, DOI 10.1080/00927872.2017.1378894, \url{https://doi.org/10.1080/00927872.2017.1378894}.
\item Pal, Chandan and Pradhan, Somnath: Risk sensitive control of pure jump processes on a general
state space, \emph{Stochastics} {\bf 91} (2019), 155--174, DOI 10.1080/17442508.2018.1521413, \url{https://doi.org/10.1080/17442508.2018.1521413}.
\item Pal, Ratna and Verma, Kaushal: Ergodic properties of families of {H}\'{e}non maps, \emph{Ann. Polon. Math.} {\bf 121} (2018), 45--71, DOI 10.4064/ap170928-2-4, \url{https://doi.org/10.4064/ap170928-2-4}.
\item Paul, Kallol and Sain, Debmalya: Birkhoff-{J}ames orthogonality and its application in the
study of geometry of {B}anach space, \emph{Advanced topics in mathematical analysis} {\bf } (2019), 245--284.
\item Paul, Kallol and Sain, Debmalya and Ghosh, Puja: Symmetry of {B}irkhoff-{J}ames orthogonality of bounded linear
operators, \emph{Ulam type stability} {\bf } (2019), 331--344.
\item Paul, Kallol and Sain, Debmalya and Mal, Arpita: Approximate {B}irkhoff-{J}ames orthogonality in the space of
bounded linear operators, \emph{Linear Algebra Appl.} {\bf 537} (2018), 348--357, DOI 10.1016/j.laa.2017.10.008, \url{https://doi.org/10.1016/j.laa.2017.10.008}.
\item Paul, Kallol and Sain, Debmalya and Mal, Arpita and Mandal,
Kalidas: Orthogonality of bounded linear operators on complex {B}anach
spaces, \emph{Adv. Oper. Theory} {\bf 3} (2018), 699--709, DOI 10.15352/aot.1712-1268, \url{https://doi.org/10.15352/aot.1712-1268}.
\item Pingali, Vamsi P.: A note on the deformed {H}ermitian {Y}ang-{M}ills {PDE}, \emph{Complex Var. Elliptic Equ.} {\bf 64} (2019), 503--518, DOI 10.1080/17476933.2018.1454914, \url{https://doi.org/10.1080/17476933.2018.1454914}.
\item Pingali, Vamsi Pritham: A note on higher extremal metrics, \emph{Trans. Amer. Math. Soc.} {\bf 370} (2018), 6995--7010, DOI 10.1090/tran/7416, \url{https://doi.org/10.1090/tran/7416}.
\item Pingali, Vamsi Pritham: Representability of {C}hern-{W}eil forms, \emph{Math. Z.} {\bf 288} (2018), 629--641, DOI 10.1007/s00209-017-1903-2, \url{https://doi.org/10.1007/s00209-017-1903-2}.
\item Pritham Pingali, Vamsi: Existence of coupled {K}\"{a}hler-{E}instein metrics using the
continuity method, \emph{Internat. J. Math.} {\bf 29} (2018), 1850041, 8, DOI 10.1142/S0129167X18500416, \url{https://doi.org/10.1142/S0129167X18500416}.
\item Quddus, Safdar: Invariants of the {$\Bbb Z_2$} orbifolds of the {P}odle\'{s} two
spheres, \emph{J. Noncommut. Geom.} {\bf 13} (2019), 257--267, DOI 10.4171/JNCG/320, \url{https://doi.org/10.4171/JNCG/320}.
\item Reddy, Nanda Kishore: Equality of {L}yapunov and stability exponents for products of
isotropic random matrices, \emph{Int. Math. Res. Not. IMRN} {\bf } (2019), 606--624, DOI 10.1093/imrn/rnx134, \url{https://doi.org/10.1093/imrn/rnx134}.
\item Reza, Md. Ramiz: Curvature inequalities for operators in the {C}owen-{D}ouglas
class of a planar domain, \emph{Indiana Univ. Math. J.} {\bf 67} (2018), 1255--1279, DOI 10.1512/iumj.2018.67.7320, \url{https://doi.org/10.1512/iumj.2018.67.7320}.
\item Roth, Julien and Upadhyay, Abhitosh: On compact anisotropic {W}eingarten hypersurfaces in
{E}uclidean space, \emph{Arch. Math. (Basel)} {\bf 113} (2019), 213--224, DOI 10.1007/s00013-019-01315-8, \url{https://doi.org/10.1007/s00013-019-01315-8}.
\item Roy, Krishanu and Venkatesh, R.: Maximal closed subroot systems of real affine root systems, \emph{Transform. Groups} {\bf 24} (2019), 1261--1308, DOI 10.1007/s00031-018-9510-9, \url{https://doi.org/10.1007/s00031-018-9510-9}.
\item Sain, Debmalya: On extreme contractions and the norm attainment set of a
bounded linear operator, \emph{Ann. Funct. Anal.} {\bf 10} (2019), 135--143, DOI 10.1215/20088752-2018-0014, \url{https://doi.org/10.1215/20088752-2018-0014}.
\item Sain, Debmalya: On the norm attainment set of a bounded linear operator, \emph{J. Math. Anal. Appl.} {\bf 457} (2018), 67--76, DOI 10.1016/j.jmaa.2017.07.070, \url{https://doi.org/10.1016/j.jmaa.2017.07.070}.
\item Sain, Debmalya: Smooth points in operator spaces and some
{B}ishop-{P}helps-{B}ollob\'{a}s type theorems in {B}anach spaces, \emph{Oper. Matrices} {\bf 13} (2019), 433--445, DOI 10.7153/oam-2019-13-32, \url{https://doi.org/10.7153/oam-2019-13-32}.
\item Sain, Debmalya and Paul, Kallol and Bhunia, Pintu and Bag,
Santanu: On the numerical index of polyhedral {B}anach spaces, \emph{Linear Algebra Appl.} {\bf 577} (2019), 121--133, DOI 10.1016/j.laa.2019.04.024, \url{https://doi.org/10.1016/j.laa.2019.04.024}.
\item Sain, Debmalya and Paul, Kallol and Mal, Arpita: A complete characterization of {B}irkhoff-{J}ames
orthogonality in infinite dimensional normed space, \emph{J. Operator Theory} {\bf 80} (2018), 399--413.
\item Sain, Debmalya and Paul, Kallol and Mal, Arpita: On approximate {B}irkhoff-{J}ames orthogonality and normal
cones in a normed space, \emph{J. Convex Anal.} {\bf 26} (2019), 341--351.
\item Sain, Debmalya and Paul, Kallol and Mandal, Kalidas: On two extremum problems related to the norm of a bounded
linear operator, \emph{Oper. Matrices} {\bf 13} (2019), 421--432, DOI 10.7153/oam-2019-13-31, \url{https://doi.org/10.7153/oam-2019-13-31}.
\item Sain, Debmalya and Ray, Anubhab and Paul, Kallol: Extreme contractions on finite-dimensional polygonal {B}anach
spaces, \emph{J. Convex Anal.} {\bf 26} (2019), 877--885.
\item Sanki, Bidyut: Filling of closed surfaces, \emph{J. Topol. Anal.} {\bf 10} (2018), 897--913, DOI 10.1142/S1793525318500309, \url{https://doi.org/10.1142/S1793525318500309}.
\item Sanki, Bidyut and Gadgil, Siddhartha: Graphs of systoles on hyperbolic surfaces, \emph{J. Topol. Anal.} {\bf 11} (2019), 1--20, DOI 10.1142/S1793525319500018, \url{https://doi.org/10.1142/S1793525319500018}.
\item Seshadri, Harish: Differential geometry in {I}ndia, \emph{Indian J. Pure Appl. Math.} {\bf 50} (2019), 795--799, DOI 10.1007/s13226-019-0355-2, \url{https://doi.org/10.1007/s13226-019-0355-2}.
\item Seshadri, Harish and Verma, Kaushal: Some aspects of the automorphism groups of domains, \emph{Handbook of group actions. {V}ol. {III}} {\bf 40} (2018), 145--174.
\item Thangavelu, Sundaram: An analogue of {P}fannschmidt's theorem for the {H}eisenberg
group, \emph{J. Anal.} {\bf 26} (2018), 235--244, DOI 10.1007/s41478-018-0147-9, \url{https://doi.org/10.1007/s41478-018-0147-9}.
\item Thangavelu, Sundaram and Naidu Dogga, Venku: {$L^p$}-{$L^2$} estimates for solutions of the wave equation
associated to the {G}rushin operator, \emph{Adv. Pure Appl. Math.} {\bf 9} (2018), 85--92, DOI 10.1515/apam-2017-0042, \url{https://doi.org/10.1515/apam-2017-0042}.
\item Turgay, Nurettin Cenk and Upadhyay, Abhitosh: On biconservative hypersurfaces in 4-dimensional {R}iemannian
space forms, \emph{Math. Nachr.} {\bf 292} (2019), 905--921, DOI 10.1002/mana.201700328, \url{https://doi.org/10.1002/mana.201700328}.
\item Urbina-Romero, Wilfredo: Gaussian harmonic analysis, \emph{} {\bf } (2019), xix+477, DOI 10.1007/978-3-030-05597-4, \url{https://doi.org/10.1007/978-3-030-05597-4}.
\item Verma, Kaushal: Notes on the boundaries of quadrature domains, \emph{Anal. Math. Phys.} {\bf 9} (2019), 617--638, DOI 10.1007/s13324-018-0221-0, \url{https://doi.org/10.1007/s13324-018-0221-0}.
\item Vishnu Priya, N. and Senthilvelan, M. and Rangarajan,
Govindan: On the role of four-wave mixing effect in the interactions
between nonlinear modes of coupled generalized nonlinear
{S}chr\"{o}dinger equation, \emph{Chaos} {\bf 29} (2019), 123135, 14, DOI 10.1063/1.5121245, \url{https://doi.org/10.1063/1.5121245}.
\item Vishnu Priya, N. and Senthilvelan, M. and Rangarajan, Govindan
and Lakshmanan, M.: On symmetry preserving and symmetry broken bright, dark and
antidark soliton solutions of nonlocal nonlinear {S}chr\"{o}dinger
equation, \emph{Phys. Lett. A} {\bf 383} (2019), 15--26, DOI 10.1016/j.physleta.2018.10.011, \url{https://doi.org/10.1016/j.physleta.2018.10.011}.
\end{enumerate}

\subsection{List of Visitors to the Department}

\begin{enumerate}
\item Dr.R. Venkatesh, (a) Settled the decomposition theorem for G-stable affine Demazure modules for the cases $G = E_6,7,8$ and $F_4$ and thus completed the main theorem of the paper of V. Chari and et al., (J. Algebra, 2016). Obtained a new combinatorial proof for the key fact that was used in  V. Chari and et al., (J. Algebra, 2016) to prove the decomposition of G-stable affine Demazure modules. This key fact was proved by V. Chari and et al., (J. Algebra, 2016) only for the cases when G is the classical types and $G_2$ and they did a case-by-case analysis to prove this. Our new proof is very uniform and avoids the case-by-case analysis and works in general.  (b) Simplified the presentation of all affine Demazure modules of untwisted affine Kac–Moody Lie algebras. Simplifying the generators and relations of these modules is very crucial in their study. We are investigating how to use this simplified presentation of affine Demazure modules to study Schur positivity type questions which arise naturally in Algebraic Combinatorics.   (c) A number of conjectures made about the graded characters affine Demazure modules of affine $sl_3$ using data obtained from SAGE. , --
\item Dr.Govindan Rangarajan, We constructed [1] symmetry preserving and symmetry broken N-bright, dark and antidark soliton solutions of a nonlocal nonlinear Schrödinger equation. To obtain these solutions, we used appropriate eigenfunctions in Darboux transformation (DT) method. Further, we derived two dark/antidark soliton solution with the help of DT method. In the dark/antidark soliton solution case we observed a contrasting behaviour between the envelope of the field and parity transformed complex conjugate envelope of the field. For a particular parametric choice, we got dark (antidark) soliton for the field while the parity transformed complex conjugate field exhibits antidark (dark) soliton. Due to this surprising result, both the field and PT transformed complex conjugate field exhibit sixteen different combinations of collision scenario.   We presented observational evidence [2] from studies on primary cortical cultures from AD transgenic mice for significant decrease in total spine density at DIV-15 and onward. This indicates reduction in potential healthy synapses and strength of connections among neurons. Based on this, a network model of neurons was developed, that explains the consequent loss of coordinated activity and transmission efficiency among neurons that manifests over time. The critical time when structural connectivity in the brain undergoes a phase-transition, from initial robustness to irreparable breakdown, is estimated from this model. We also showed how the global efficiency of signal transmission in the network decreases over time. Moreover, the number of multiple paths of high efficiency decreased rapidly as the disease progresses, indicating loss of structural plasticity and inefficiency in choosing alternate paths or desired paths for any pattern of activity. Thus loss of spines caused by β-Amyloid (Aβ) peptide results in disintegration of the neuronal network over time with consequent cognitive dysfunctions in Alzheimer’s Disease (AD).  We investigated [3] the effect of four-wave mixing in the interactions among nonlinear waves such as solitons, breathers, and rogue waves of a coupled generalized nonlinear Schrodinger equation. We explored several interesting results including superposition of breather pulses, increment in the number of breather pulses and in amplitudes of breathers, and rogue waves. By strengthening the four-wave mixing parameter, we observed different transformations that occur between different localized structures. For instance, we visualized a transformation from bright soliton to breather form, bright and dark rogue wave to four-petaled rogue wave structures, four-petaled rogue wave to other rogue wave forms, and so on. Another important observation was that the interaction of a bright soliton with a rogue wave in the presence of the four-wave mixing effect provides interaction between a dark oscillatory soliton and a rogue wave.  [1] Vishnu Priya, N., Senthilvelan, M.,  Rangarajan, G. & Lakshmanan, M. On symmetry preserving and symmetry broken bright, dark and antidark soliton solutions of nonlocal nonlinear Schrödinger equation, Physics Letters A, 383, 15 (2019).   [2] Kashyap, G. A. R. S. R. K., Bapat, D., Das, D., Gowaikar, R. D., Amritkar, R. E., Rangarajan, G., Ravindranath, V. & Ambika, G., Synapse loss and progress of Alzheimer’s disease - A network model, Scientific Reports, 9, 6555 (2019).   [3] Vishnu Priya, N., Senthilvelan, M. &  Rangarajan, G, On the role of four-wave mixing effect in the interactions between nonlinear modes of coupled generalized nonlinear Schrodinger equation, Chaos, 29, 123135 (2019)., On symmetry preserving and symmetry broken bright, dark and antidark soliton solutions of nonlocal nonlinear Schrödinger equation--Published
\item Dr.Ved Datar, In collaboration with Vamsi Pingali [1],  a numerical criteria for the existence of solutions to generalised Monge-Ampere equations is obtained  on projective manifolds. The class of equations studied encompasses certain well known equations such as the $J$-equation and the inverse Hessian equations. In particular, this extends and strengthens the deep result of Gao Chen on the $J$-equation, and also resolves a well-known conjecture of Gabor Szekelyhidi.   In a collaboration with Xin Fu and Jian Song [2], Kahler-Einstein metrics were constructed in the neighbourhoods of isolated log-canonical singularities, and the asymptotics to certain local models were studied. This builds on a program initiated by Kobayashi and others.    In a joint project with Harish Seshadri, a conjecture on the almost rigidity of the complex projective space has been formulated, and is being studied. If solved, this would be a Kahler analog of a famous theorem of Tobias Colding on the almost rigidity of the sphere. While some preliminary results have been obtained, this is still work in progress.  [1] (with Vamsi Pingali) A numerical criterion for generalised Monge-Ampere equations on projective manifolds, arXiv:2006.01530 [2] (with Xin Fu and Jian Song) On the Kahler-Einstein metric near isolated log-canonical singularity,  in  preparation, Adiabatic limits of anti-self-dual connections on collapsed K3 surfaces--Accepted for publication
\item Dr.Vamsi Pritham Pingali, My research during this period involved proving existence results for geometric PDE conditioned on some necessary and sufficient conditions. In [1], the dHYM equation on three-folds was studied, and in [2] we proved an existence result for the gravitating vortex equation conditioned on a GIT stability condition. In addition, some earlier works [3],[4],[5],[6] were accepted/published during this period.  [1] Pingali, Vamsi Pritham. "The deformed Hermitian Yang-Mills equation on three-folds." arXiv preprint arXiv:1910.01870 (2019). [2] Garcia-Fernandez, Mario, Vamsi Pritham Pingali, and Chengjian Yao. "Gravitating vortices with positive curvature." arXiv preprint arXiv:1911.09616 (2019). [3] PINGALI, V., & VAROLIN, D. (2019). NONUNIFORMLY FLAT AFFINE ALGEBRAIC HYPERSURFACES. Nagoya Mathematical Journal, 1-43. doi:10.1017/nmj.2019.2 (Published online on 2 April  2019) [4] Pingali, Vamsi Pritham. "A vector bundle version of the Monge-Ampère equation." Advances in Mathematics 360 (2020): 106921. (Published online on 22 January 2020) [5] Álvarez-Cónsul, Luis, Mario Garcia-Fernandez, Oscar García-Prada, and Vamsi Pritham Pingali. "Gravitating vortices and the Einstein–Bogomol’nyi equations." Mathematische Annalen (2020): 1-34. (Published online on 20 February 2020) [6] Pingali, Vamsi Pritham. "Quillen metrics and perturbed equations." Letters in Mathematical Physics (2020): 1-15. (Published online on 5 March 2020), The deformed Hermitian Yang-Mills equation on three-folds--Preprint
\item Dr.Basudeb Datta, A semi-regular tiling of the hyperbolic plane is a tessellation by regular geodesic polygons with the property that each vertex has the same vertex-type. Determined combinatorial criteria for the existence, and uniqueness, of a semi-regular tiling with a given vertex-type \ref{bdsg_2019} ., Semi-regular tilings of the hyperbolic plane \label{bdsg_2019}--Accepted for publication
\item Dr.E K Narayanan, Positivity properties and asymptotics of Heckman-Opdam hypergeometric functions associated to root system to type BC with non-positive multiplicity functions were studied in detail., --
\item Dr.Soumya Das, Work on the determination of Siegel modular forms by their fundamental Fourier coefficients was done including a mod p version. Study of Fourier coefficients and sup-norms of modular forms was continued., --
\item Dr.Dilip P. Patil, COMPUTATIONAL ASPECTS OF BURNSIDE RINGS PART II: IMPORTANT MAPS (Jointly Martin  Kreuzer)   The Burnside ring B(G) of a finite group G, a classical tool in the representation theory of finite groups, is studied from the point of view of computational algebra. In the rst part (cf. [8]) we examined the ring theoretic properties of B(G) using the methods of computer algebra. In this part we shift our focus to important maps between two Burnside rings and make several well-known maps of representation theory explicitly computable. The inputs of all our algorithms are the tables of marks of the two groups, and the outputs are matrices of integers representing the maps via their image in the ghost ring. All algorithms have been implemented in the computer algebra system ApCoCoA and are illustrated by applying them to explicit examples. Especially, we study the restriction and induction maps, the projection and in ation maps, the conjugation isomorphism, and the Frobenius-Wielandt homomorphism., COMPUTATIONAL ASPECTS OF BURNSIDE RINGS PART II: IMPORTANT MAPS--Accepted for publication
\item Dr.A. K. Nandakumaran, The main theme of the research is to understand the asymptotic analysis of  PDE problems and related optimal control problems in some complex domains. We consider domains whose boundary is rapidly oscillating which  leads to multi-scales in the problem. One of the major issue is the development of the so called unfolding operators which contains information about both scales. We have ingeniously used the unfolding operators not only to carry out the limiting behaviour of the problem under study, it is used to characterize the optimal controls. One needs to construct suitable such operators according to the type of domain and the problem [1], [3-5]. On a completely different direction, a multi-step Volterra integral equation-based algorithm was developed to measure the electric field auto-correlation function from multi-exposure speckle contrast data. This enabled us to derive an estimate of the full diffuse correlation spectroscopy data-type from a low-cost, camera-based system. This method is equally applicable for both single and multiple scattering field auto-correlation models. The feasibility of the system and method was verified using simulation studies, tissue mimicking phantoms and subsequently in in vivo experiments [2]., Asymptotic analysis of Boundary Optimal Control Problem on a General Branched Structure--Published
\item Dr.Gautam Bharali, A detailed investigation of the Bergman geometry of domains whose boundaries have isolated infinite-type points was completed. Optimal estimates, from above and below, for the growth of the Bergman kernel (on the diagonal) and the Bergman metric, as one approaches an infinite-type boundary point, were obtained. The class of domains that were studied subsumes all classes of domains that have previously been considered in the literature. These results have appeared in the paper [1].  [1] Bharali, Gautam, On the growth of the Bergman metric near a point of infinite type, J. Geom. Anal. , 30 (2020), 1238-1258., A weak notion of visibility, a family of examples, and Wolff--Denjoy theorems--Accepted for publication
\item Dr.SANCHAYAN SEN, In [1], we provide an explicit algorithm for sampling a uniform simple connected random graph with a given degree sequence. By products of this central result includes scaling limits for the metric space structure of the maximal components in the critical regime of both the configuration model and the uniform simple random graph model with prescribed degree sequence under finite third moment assumption on the degree sequence. As a substantive application we answer a question raised by Černý and Teixeira, Random Structures Algorithms (2013), by obtaining the metric space scaling limit of maximal components in the vacant set left by random walks on random regular graphs.  In [2], we study limits of the largest connected components (viewed as metric spaces) obtained by critical percolation on uniformly chosen graphs and configuration models with heavy-tailed degrees. We develop general principles under which the scaling limits can be obtained. Of independent interest, we derive refined asymptotics for various susceptibility functions and the maximal diameter in the barely subcritical regime.  In [3], we give alternate constructions of (i) the scaling limit of the uniform connected graphs with given fixed surplus, and (ii) the continuum random unicellular map of a given genus that start with a suitably tilted Brownian continuum random tree and make `horizontal' point identifications, at random heights, using the local time measures. In particular, this yields a breadth-first construction of the scaling limit of the critical Erdős-Rényi random graph which answers a question posed by Addario-Berry, Broutin, and Goldschmidt, Probab. Theory Related Fields (2012). As a consequence of this breadth-first construction we obtain descriptions of the radii, the distance profiles, and the two point functions of these spaces in terms of functionals of tilted Brownian excursions.    [1] Geometry of the vacant set left by random walk on random graphs, Wright's constants, and critical random graphs with prescribed degrees Authors: Shankar Bhamidi, Sanchayan Sen Journal: Random Structures & Algorithms (First published on 25 July 2019)  [2] Universality for critical heavy-tailed network models: Metric structure of maximal components Authors: Shankar Bhamidi, Souvik Dhara, Remco van der Hofstad, Sanchayan Sen Journal: Electronic Journal of Probability (Accepted in December 2019)  [3] On breadth-first constructions of scaling limits of random graphs and random unicellular maps Authors: Grégory Miermont, Sanchayan Sen ArXiv preprint https://arxiv.org/abs/1908.04403 (Announced in August 2019), On breadth-first constructions of scaling limits of random graphs and random unicellular maps--Submitted
\item Dr.Arvind Ayyer, In joint work with Aas, Linusson and Potka [1], the exact phase diagram for a semipermeable TASEP with nonlocal boundary jumps was obtained. In joint work with Behrend [2], new factorization theorems for classical group characters were proved. Applications to alternating sign matrices and plane partitions were given. In joint work with Finn and Roy [3], the phase diagram for a multispecies left-permeable asymmetric exclusion process was derived. In joint work with Linusson [4], a new class of reverse juggling processes was defined and their stationary distributions were computed., Total Variation Cutoff for the Transpose Top-2 with Random Shuffle--Accepted for publication
\item Dr.Subhojoy Gupta, (1) With Mahan Mj (TIFR)  we completed a  research program concerning meromorphic projective structures on surfaces and their monodromy representations. We used new  work of Allegretti-Bridgeland, together with older ideas of Thurston, to develop a geometric understanding of these structures on punctured surfaces, and prove analogues of results of Hejhal and Gallo-Kapovich-Marden -- see [1] and [2]. In an independent paper [3], I used our the techniques we had developed, to shed light on a remark of Poincaré in 1883. This complemented other projects concerning meromorphic quadratic differentials (see [8] and [10]).  (2) In joint work [4] with Indranil Biswas (TIFR), Mahan Mj (TIFR) and Junho Whang (MIT), we classified the finite mapping class group orbits in the PSL(2,C) character variety.  (3) In a project with Basudeb Datta (IISc) we revised and published a paper [5] developing an algorithmic technique of constructing semi-regular tilings of the hyperbolic plane. We expect this to have an impact on tilings-related research, which is an active area that has an intersection with several fields like combinatorics, computational geometry and group theory.   (4)  With Harish Seshadri (IISc) we wrote a paper [6], and a survey [7], concerning the complex geometry of Teichmüller spaces, where we investigated the convexity (or lack thereof) of the boundary of complex domains biholomorphic to Teichmüller space. We have ongoing conversations regarding other questions that lie at the intersection of several complex variables, differential geometry and Teichmüller theory.   (5) A paper with Indranil Biswas (TIFR) and Sorin Dumitrescu (UCA)  concerning branched projective structures on surfaces was published [9]. In an ongoing project, we aim to understand the symplectic geometry of the space of such structures better.  [1] Meromorphic projective structures, grafting and the monodromy map, submitted, available at http://arxiv.org/abs/1904.03804. [2] Monodromy representations of meromorphic projective structures, Proc. Amer. Math. Soc. 148 (2020), no. 5, 2069–2078. [3] Monodromy groups of CP1-structures on punctured surfaces, submitted, available at https: //arxiv.org/abs/1909.10771. [4] Surface group representations in SL(2,C) with finite mapping class orbits, submitted, available at https://arxiv.org/ abs/1707.00071. [5] Semi-regular tilings of the hyperbolic plane, to appear, Discrete and Computational Geometry (2019), available at https://doi.org/10.1007/s00454-019-00156-0. [6] On domains biholomorphic to Teichmüller spaces, International Mathematics Research Notices, Volume 2020, Issue 8, (2020), 2542–2560. [7] Complex geometry of Teichmüller domains, Handbook of Teichmüller theory Vol VII, 63–88, IRMA Lect. Math. Theor. Phys., 30, Eur. Math. Soc., Zürich, 2020. [8] Holomorphic quadratic differentials in Teichmüller theory, Handbook of Teichmü ller theory Vol VII, 2020. [9] Branched projective structures on a Riemann surface and logarithmic connections, Documenta Mathematica 24 (2019), 2299–2337. [10] Limits of harmonic maps and crowned hyperbolic surfaces, Trans. Amer. Math. Soc. 372 (2019), no. 11, 7573–7596., Meromorphic projective structures, grafting and the monodromy map--Preprint
\item Dr.Mahesh Kakde, Jointly with Samit Dasgupta from Duke University I resolved the Brumer-Stark conjecture during the said period. We spoke about it at the "International Colloquium" in January 2020 in TIFR. Further, a paper on higher Chern classes written a while ago (jointly with Frauke Bleher, Ted Chinburg,  Ralph Greenberg, George Pappas, Romyar Sharifi and Martin Taylor) was accepted for publication in American Journal of Mathematics., Higher Chern Classes in Iwasawa Theory--Published
\item Dr.RADHIKA GANAPATHY, 1) Let $G$ be a split connected reductive group over $\mathbb{Z}$. Let $F$ be a non-archimedean local field. With $K_m := Ker(G(O_F ) → G(O_F /p_{m,F} ))$, Kazhdan proved that for a field $F′$ sufficiently close to $F$, the Hecke algebras $\mathcal{H}(G(F), K_m)$ and $\mathcal{H}(G(F′), K_m′ )$ are isomorphic, where $K_m′$ denotes the corresponding object over $F′$. I have generalized this result to general connected reductive groups [1].  2) In my earlier work, I had established  a variant of the Kazhdan isomorphism for split reductive groups (by replacing $K_m$ replaced by the $m$-th filtration subgroup of the Iwahori subgroup). I have been working on generalising this result to general connected reductive groups. As part of this work, I have chosen a nice set of representatives in $G(F)$ for the elements of the extended affine Weyl group. This set of representatives has some very interesting properties (for instance, it is compatible with unramified descent), and is useful in the study of representation theory of $p$-adic groups.   [1] A Hecke algebra isomorphism over close local fields, preprint, 12 pages., A Hecke algebra isomorphism over close local fields--Preprint
\item Dr.Manjunath Krishnapur, Completed a study of law of iterated logarithm for last passage percolating on the two dimensional integer lattice [1]. Other studies on absolute continuity of limiting spectral distributions, a proof of KMT theorems, study of discrete nodal length, a problem on amplitude constrained Gaussian channel and some problems on nodal domains of random functions are in various stages of completion.  [1] Lower Deviations in β-ensembles and Law of Iterated Logarithm in Last Passage Percolation.   Authors: Riddhipratim Basu, Shirshendu Ganguly, Milind Hegde, Manjunath Krishnapur. arXiv:1909.01333, Lower Deviations in β-ensembles and Law of Iterated Logarithm in Last Passage Percolation--Accepted for publication
\item Dr.Sundaram Thangavelu, In a joint work with Pritam Ganguly  we have investigated the $ L^p $  boundedness of the lacunary maximal function $ M_{\H^n}^\lac $ associated to the spherical means $ A_r f$  taken over Koranyi spheres on the Heisenberg group. Closely following an approach used by M. Lacey in the Euclidean case, we obtain sparse bounds for these maximal functions leading to new unweighted and weighted estimates. The key ingredients in the proof are the $L^p$  improving property of the operator $ A_r $  and a continuity property of the difference $ A_r f−\tau_y A_r f,$ where $\tau_yf(x)=f(xy−1) $ is the right translation operator., On lacunary spherical maximal function on Heisenberg groups--Preprint
\item Dr.R VENKATESH, (a) Settled the decomposition theorem for G-stable affine Demazure modules for the cases $G = E_6,7,8$ and $F_4$ and thus completed the main theorems of the papers of V. Chari and et al., (J. Algebra, 2016) and Deniz Kus and et al., (Represent. Theory., 2016). Obtained a new combinatorial proof for the key fact that was used in  V. Chari and et al., (J. Algebra, 2016) to prove the decomposition of G-stable affine Demazure modules. This key fact was proved only for the cases when G is the classical type and $G_2$ in their paper and they also proved it by analysing case-by-case. Our proof is uniform avoids the case-by-case analysis and works for all finite-dimensional simple Lie algebras.  (b) Simplified the presentation of all affine Demazure modules of untwisted affine Kac–Moody Lie algebras. Simplifying the generators and relations of these modules is very crucial in their study. We are investigating how to use this simplified presentation of affine Demazure modules to study Schur positivity type questions which arise naturally in Algebraic Combinatorics.   (c) A number of conjectures made about the graded characters affine Demazure modules of affine $sl_3$ using data obtained from SAGE. , --
\item Dr.Apoorva Khare, (1) I extended a fundamental matrix inequality, the Schur Product Theorem, for positive definite matrices, by providing nonzero, tight lower bounds. (2) I improved and updated several preprints - notably, a comprehensive work titled "Moment-sequence transforms" with Belton-Guillot-Putinar., Probability inequalities and tail estimates for metric semigroups--Published
\item Dr.Siddhartha Gadgil, We continued our development of automated theorem proving systems. Specifically, we developed a framework based on equation terms to allow accumulation of knowledge and facilitate deep learning and hybrid autonomous/interactive execution., --
\item Dr.Tirthankar Bhattacharyya, 1. Distinguished varieties:  A distinguished algebraic variety in $\mathbb{C}^2$ has been the focus of much research in recent years because of good reasons. We have given a different perspective.  \begin{enumerate}  \item We have found a new characterization of an algebraic variety $\mathcal W$ which is distinguished with respect to the bidisc. It is in terms of the joint spectrum of a pair of commuting linear matrix pencils.  \item There is a characterization known of $\mathbb{D}^2\cap\mathcal{W}$ due to a seminal work of Agler and McCarthy. We have shown that Agler--McCarthy characterization can be obtained from the new one and vice versa.  \item En route, we develop a new realization formula for operator-valued contractive analytic functions on the unit disc.  \item There is a one-to-one correspondence between operator-valued contractive holomorphic functions and {\em canonical model triples}. This pertains to the new realization formula mentioned above.  \item Pal and Shalit gave a characterization of an algebraic variety, which is distinguished with respect to the symmetrized bidisc, in terms of a matrix of numerical radius no larger than $1$. We refine their result by making the class of matrices strictly smaller.  \item In a generalization in the direction of more than two variables, we have characterized all one-dimensional algebraic  varieties which are distinguished  with respect to the polydisc. \end{enumerate} At the root of our work is the Berger--Coburn--Lebow theorem characterizing a commuting tuple of isometries. This work is in [1].  2. We have studied the action of the automorphism group of the $2$ complex dimensional manifold symmetrized bidisc $\mathbb G$ on itself. The automorphism group is $3$ real dimensional. It foliates $\mathbb G$ into leaves all of which are $3$ real dimensional hypersurfaces except one, viz., the royal variety. This leads us to investigate Isaev's classification of all Kobayashi-hyperbolic $2$ complex dimensional  manifolds for which the group of holomorphic automorphisms has real dimension $3$ studied by Isaev. Indeed, we produce a biholomorphism between the symmetrized bidisc and the domain  \[\{(z_1,z_2)\in \mathbb{C} ^2 : 1+|z_1|^2-|z_2|^2>|1+ z_1 ^2 -z_2 ^2|, Im(z_1 (1+\overline{z_2}))>0\}\]  in Isaev's list. Isaev calls it $\mathcal D_1$. The road to the biholomorphism is paved with various geometric insights about $\mathbb G$.    Several consequences of the biholomorphism follow including two new characterizations of the symmetrized bidisc and several new characterizations of $\mathcal D_1$. Among the results on $\mathcal D_1$, of particular interest is the fact that $\mathcal D_1$ is a ``symmetrization''. When we symmetrize (appropriately defined in this context) either $\Omega_1$ or $\mathcal{D}^{(2)} _1$ (Isaev's notation), we get $\mathcal D_1$.  These two domains $\Omega_1$ and $\mathcal{D}^{(2)} _1$ are in Isaev's list and he mentioned that these are biholomorphic to $\mathbb D \times \mathbb D$. We produce explicit biholomorphisms between these domains and $\mathbb D \times \mathbb D$. This work is in [2].  [1]  Distinguished Varieties Through the Berger--Coburn--Lebow Theorem,  co-authored with P. Kumar and H. Sau. Submitted to journal and arXiv link is https://arxiv.org/pdf/2001.01410.pdf   [2] An unbounded realization of the symmetrized bidisc, co-authored with A. Biswas and A. Maitra. Submitted to journal and arXiv link is  https://arxiv.org/pdf/2005.00289.pdf, On the Nevanlinna problem: characterization of all Schur–Agler class solutions affiliated with a given kernel--Published
\item Dr.Purvi Gupta, Note: this description is for research carried out during the period February 11, 2020 (date of joining) to March 31, 2020.   In this period,  I completed and submitted a manuscript ([1]),  based on joint work with C. U. Wawrzyniak. This work studies the stability of the solution of the so-called global Bishop problem (a plateau-type problem) for a model real $n$-dimensional sphere in $\mathbb{C}^n$, $n\geq 2$. In this problem, the solution is required, among other things, to be foliated by holomorphic disks (i.e., Levi-flat) attached to the given sphere. This structure is significant in complex analysis for multiple reasons. The general Bishop problem, although very well-understood in two (complex) dimensions, is wide open in dimensions three and higher, and our work is the first global result in this direction.    References [1] Gupta, Purvi and Wawrzyniak, Chloe Urbanski, Stability of the hull(s) of an $n$-sphere in $\mathbb{C}^n$. Submitted., STABILITY OF THE HULL(S) OF AN $n$-SPHERE IN $\mathbb{C}^n$--Submitted
\item Dr.Kaushal Verma, For a given pair of smoothly bounded planar domains $D_1, D_2$, it was shown in \cite{1} that the product of the Carath\'{e}odory metrics on them is bounded from below by the product of the Carath\'{e}odory metrics on their union and intersection up to a multiplicative factor. This shows that the S\"{z}ego kernels on $D_1, D_2, D_1 \cap D_2$ and $D_1 \cup D_2$ are related by the same inequality.    The Suita conjecture in higher dimensions is a paradigm that asks for the limiting boundary values of the product of the Bergman kernel (on the diagonal) and the volume of the Kobayashi indicatrix. This object is a biholomorphic invariant and is known to admit universal bounds on convex domains. Its limiting behaviour was clarified on strongly peudoconvex domains and egg domains in $\mathbb C^2$. See \cite{2}.  The boundary behaviour of several classical conformal invariants were studied in \cite{3}.  The relationship between the Fridman invariant and the Squeezing function was clarified in \cite{4}.  The relation between a pair of H\'{e}non maps that share the same forward and backward Julia sets was clarified in \cite{5}., A submultiplicative property of the Carath\'{e}odory metric on planar domains--Published
\item Dr.Srikanth Krishnan Iyer, We study combinatorial connectivity for two models of random geometric complexes. These two models - \v{C}ech and Vietoris-Rips complexes - are built on a homogeneous Poisson point process of intensity $n$ on a $d$-dimensional torus, $d > 1$, using balls of radius $r_n$. Given a (simplicial) complex (i.e., a collection of $k$-simplices for all $k \geq 1$), we can connect $k$-simplices via $(k+1)$-simplices (`up-connectivity') or via $(k-1)$-simplices (`down-connectivity). Our interest is to understand these two combinatorial notions of connectivity for the random \v{C}ech and Vietoris-Rips complexes asymptotically as $n \to \infty$. Our results give tighter bounds on the constants in the logarithmic scale as well as shed light on the possible second-order correction factors. Further, they also reveal interesting differences between the phase transition in the \v{C}ech and Vietoris-Rips cases. The analysis is interesting due to non-monotonicity of the number of isolated $k$-faces (as a function of the radius) and leads one to consider `monotonic' vanishing of isolated $k$-faces. The latter coincides with the vanishing threshold mentioned above at a coarse scale (i.e., $\log n$ scale) but differs in the $\log \log n$ scale for the \v{C}ech complex with $k = 1$ in the up-connected case. For the case of up-connectivity in the Vietoris-Rips complex and for $r_n$ in the critical window, we also show a Poisson convergence for the number of isolated $k$-faces when $k \leq d$. [1], Thresholds for vanishing of isolated faces in random Cech and Vietoris-Rips Complexes--Published
\item Dr.Thirupathi Gudi, We have developed a Hybrid High Order methods for elliptic obstacle problem in both two and three space dimensions allowing general meshes. The gap in literature of developing an optimal order convergent method using quadratic finite element method is resolved under the realistic regularity. In another work, we derived optimal order finite element analysis for constrained Dirichlet boundary optimal control problem. We have also studied numerous finite element methods for parabolic obstacle problem under the realistic regularity., Finite Element Analysis of the Constrained Dirichlet Boundary Control Problem Governed by the Diffusion Problem--Accepted for publication
\item Dr.Hrish Seshadri, , Professor--
\item Dr.Mrinal Kanti Ghosh, Risk-sensitive stochastic differential games have been studied in the nonnegative orthant [1] and the whole space [2]. Existence of saddle point/Nash equilibria for relevant cases have been derived., Zero-Sum Risk-Sensitive Stochastic Differential Games with Reflecting Diffusions in the Orthant--Accepted for publication
\item Dr.Gadadhar Misra, There is a one to one correspondence between holomorphic imprimitivities and the holomomorphic hermitian homogeneous vector bundles on a bounded symmetric domain D. The homogeneous bundles can be obtained by holomorphic induction from representations of a certain parabolic Lie algebra on finite dimensional inner product spaces. The representations, and the induced bundles, have composition series with irreducible factors. In joint work with A. Koranyi, our first main result is the construction of an explicit differential operator intertwining the bundle with the direct sum of its factors. Next, we study Hilbert spaces of sections of these bundles. We use this to get, in particular, a full description and a similarity theorem for homogeneous n-tuples of operators in the Cowen-Douglas class of the Euclidean unit ball.   Beurling’s theorem shows that all submodules of the Hardy module are isomorphic leading to the question of rigidity of submodules in multi-variate operator theory. The submodules of analytic Hilbert modules defined over certain algebraic varieties in bounded symmetric domains, the so-called Jordan-Kepler varieties X of arbitrary rank r have been studied. For r > 1, the singular set of X is not a complete intersection. Hence the usual monoidal transformations do not suffice for the resolution of the singularities. Instead (joint with Upmeier), we describe a new higher rank version of the blow-up process, defined in terms of Jordan algebraic determinants, and apply this resolution to obtain the rigidity of the sub-modules vanishing on the singular set.   (with Koranyi) Homogeneous Hermitian holomorphic vector bundles and the Cowen-Douglas class over bounded symmetric domains, Adv. Math., 351 (2019), 1105 - 1138.  (with Upmeier)  Singular Hilbert modules on Jordan-Kepler varieties, To appear, Proceedings of IWOTA 2018, in memory of R. G. Douglas., Professor--Published
\item Dr.Abhishek Banerjee, , Cohomology of modules over H-categories and co-H-categories--Accepted for publication
\end{enumerate}

